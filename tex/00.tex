
% Default to the notebook output style

    


% Inherit from the specified cell style.




    
\documentclass{article}

    
    
    \usepackage{graphicx} % Used to insert images
    \usepackage{adjustbox} % Used to constrain images to a maximum size 
    \usepackage{color} % Allow colors to be defined
    \usepackage{enumerate} % Needed for markdown enumerations to work
    \usepackage{geometry} % Used to adjust the document margins
    \usepackage{amsmath} % Equations
    \usepackage{amssymb} % Equations
    \usepackage{eurosym} % defines \euro
    \usepackage[mathletters]{ucs} % Extended unicode (utf-8) support
    \usepackage[utf8x]{inputenc} % Allow utf-8 characters in the tex document
    \usepackage{fancyvrb} % verbatim replacement that allows latex
    \usepackage{grffile} % extends the file name processing of package graphics 
                         % to support a larger range 
    % The hyperref package gives us a pdf with properly built
    % internal navigation ('pdf bookmarks' for the table of contents,
    % internal cross-reference links, web links for URLs, etc.)
    \usepackage{hyperref}
    \usepackage{longtable} % longtable support required by pandoc >1.10
    \usepackage{booktabs}  % table support for pandoc > 1.12.2
    

    
    
    \definecolor{orange}{cmyk}{0,0.4,0.8,0.2}
    \definecolor{darkorange}{rgb}{.71,0.21,0.01}
    \definecolor{darkgreen}{rgb}{.12,.54,.11}
    \definecolor{myteal}{rgb}{.26, .44, .56}
    \definecolor{gray}{gray}{0.45}
    \definecolor{lightgray}{gray}{.95}
    \definecolor{mediumgray}{gray}{.8}
    \definecolor{inputbackground}{rgb}{.95, .95, .85}
    \definecolor{outputbackground}{rgb}{.95, .95, .95}
    \definecolor{traceback}{rgb}{1, .95, .95}
    % ansi colors
    \definecolor{red}{rgb}{.6,0,0}
    \definecolor{green}{rgb}{0,.65,0}
    \definecolor{brown}{rgb}{0.6,0.6,0}
    \definecolor{blue}{rgb}{0,.145,.698}
    \definecolor{purple}{rgb}{.698,.145,.698}
    \definecolor{cyan}{rgb}{0,.698,.698}
    \definecolor{lightgray}{gray}{0.5}
    
    % bright ansi colors
    \definecolor{darkgray}{gray}{0.25}
    \definecolor{lightred}{rgb}{1.0,0.39,0.28}
    \definecolor{lightgreen}{rgb}{0.48,0.99,0.0}
    \definecolor{lightblue}{rgb}{0.53,0.81,0.92}
    \definecolor{lightpurple}{rgb}{0.87,0.63,0.87}
    \definecolor{lightcyan}{rgb}{0.5,1.0,0.83}
    
    % commands and environments needed by pandoc snippets
    % extracted from the output of `pandoc -s`
    \DefineVerbatimEnvironment{Highlighting}{Verbatim}{commandchars=\\\{\}}
    % Add ',fontsize=\small' for more characters per line
    \newenvironment{Shaded}{}{}
    \newcommand{\KeywordTok}[1]{\textcolor[rgb]{0.00,0.44,0.13}{\textbf{{#1}}}}
    \newcommand{\DataTypeTok}[1]{\textcolor[rgb]{0.56,0.13,0.00}{{#1}}}
    \newcommand{\DecValTok}[1]{\textcolor[rgb]{0.25,0.63,0.44}{{#1}}}
    \newcommand{\BaseNTok}[1]{\textcolor[rgb]{0.25,0.63,0.44}{{#1}}}
    \newcommand{\FloatTok}[1]{\textcolor[rgb]{0.25,0.63,0.44}{{#1}}}
    \newcommand{\CharTok}[1]{\textcolor[rgb]{0.25,0.44,0.63}{{#1}}}
    \newcommand{\StringTok}[1]{\textcolor[rgb]{0.25,0.44,0.63}{{#1}}}
    \newcommand{\CommentTok}[1]{\textcolor[rgb]{0.38,0.63,0.69}{\textit{{#1}}}}
    \newcommand{\OtherTok}[1]{\textcolor[rgb]{0.00,0.44,0.13}{{#1}}}
    \newcommand{\AlertTok}[1]{\textcolor[rgb]{1.00,0.00,0.00}{\textbf{{#1}}}}
    \newcommand{\FunctionTok}[1]{\textcolor[rgb]{0.02,0.16,0.49}{{#1}}}
    \newcommand{\RegionMarkerTok}[1]{{#1}}
    \newcommand{\ErrorTok}[1]{\textcolor[rgb]{1.00,0.00,0.00}{\textbf{{#1}}}}
    \newcommand{\NormalTok}[1]{{#1}}
    
    % Define a nice break command that doesn't care if a line doesn't already
    % exist.
    \def\br{\hspace*{\fill} \\* }
    % Math Jax compatability definitions
    \def\gt{>}
    \def\lt{<}
    % Document parameters
    
    
    

    % Pygments definitions
    
\makeatletter
\def\PY@reset{\let\PY@it=\relax \let\PY@bf=\relax%
    \let\PY@ul=\relax \let\PY@tc=\relax%
    \let\PY@bc=\relax \let\PY@ff=\relax}
\def\PY@tok#1{\csname PY@tok@#1\endcsname}
\def\PY@toks#1+{\ifx\relax#1\empty\else%
    \PY@tok{#1}\expandafter\PY@toks\fi}
\def\PY@do#1{\PY@bc{\PY@tc{\PY@ul{%
    \PY@it{\PY@bf{\PY@ff{#1}}}}}}}
\def\PY#1#2{\PY@reset\PY@toks#1+\relax+\PY@do{#2}}

\expandafter\def\csname PY@tok@gd\endcsname{\def\PY@tc##1{\textcolor[rgb]{0.63,0.00,0.00}{##1}}}
\expandafter\def\csname PY@tok@gu\endcsname{\let\PY@bf=\textbf\def\PY@tc##1{\textcolor[rgb]{0.50,0.00,0.50}{##1}}}
\expandafter\def\csname PY@tok@gt\endcsname{\def\PY@tc##1{\textcolor[rgb]{0.00,0.27,0.87}{##1}}}
\expandafter\def\csname PY@tok@gs\endcsname{\let\PY@bf=\textbf}
\expandafter\def\csname PY@tok@gr\endcsname{\def\PY@tc##1{\textcolor[rgb]{1.00,0.00,0.00}{##1}}}
\expandafter\def\csname PY@tok@cm\endcsname{\let\PY@it=\textit\def\PY@tc##1{\textcolor[rgb]{0.25,0.50,0.50}{##1}}}
\expandafter\def\csname PY@tok@vg\endcsname{\def\PY@tc##1{\textcolor[rgb]{0.10,0.09,0.49}{##1}}}
\expandafter\def\csname PY@tok@m\endcsname{\def\PY@tc##1{\textcolor[rgb]{0.40,0.40,0.40}{##1}}}
\expandafter\def\csname PY@tok@mh\endcsname{\def\PY@tc##1{\textcolor[rgb]{0.40,0.40,0.40}{##1}}}
\expandafter\def\csname PY@tok@go\endcsname{\def\PY@tc##1{\textcolor[rgb]{0.53,0.53,0.53}{##1}}}
\expandafter\def\csname PY@tok@ge\endcsname{\let\PY@it=\textit}
\expandafter\def\csname PY@tok@vc\endcsname{\def\PY@tc##1{\textcolor[rgb]{0.10,0.09,0.49}{##1}}}
\expandafter\def\csname PY@tok@il\endcsname{\def\PY@tc##1{\textcolor[rgb]{0.40,0.40,0.40}{##1}}}
\expandafter\def\csname PY@tok@cs\endcsname{\let\PY@it=\textit\def\PY@tc##1{\textcolor[rgb]{0.25,0.50,0.50}{##1}}}
\expandafter\def\csname PY@tok@cp\endcsname{\def\PY@tc##1{\textcolor[rgb]{0.74,0.48,0.00}{##1}}}
\expandafter\def\csname PY@tok@gi\endcsname{\def\PY@tc##1{\textcolor[rgb]{0.00,0.63,0.00}{##1}}}
\expandafter\def\csname PY@tok@gh\endcsname{\let\PY@bf=\textbf\def\PY@tc##1{\textcolor[rgb]{0.00,0.00,0.50}{##1}}}
\expandafter\def\csname PY@tok@ni\endcsname{\let\PY@bf=\textbf\def\PY@tc##1{\textcolor[rgb]{0.60,0.60,0.60}{##1}}}
\expandafter\def\csname PY@tok@nl\endcsname{\def\PY@tc##1{\textcolor[rgb]{0.63,0.63,0.00}{##1}}}
\expandafter\def\csname PY@tok@nn\endcsname{\let\PY@bf=\textbf\def\PY@tc##1{\textcolor[rgb]{0.00,0.00,1.00}{##1}}}
\expandafter\def\csname PY@tok@no\endcsname{\def\PY@tc##1{\textcolor[rgb]{0.53,0.00,0.00}{##1}}}
\expandafter\def\csname PY@tok@na\endcsname{\def\PY@tc##1{\textcolor[rgb]{0.49,0.56,0.16}{##1}}}
\expandafter\def\csname PY@tok@nb\endcsname{\def\PY@tc##1{\textcolor[rgb]{0.00,0.50,0.00}{##1}}}
\expandafter\def\csname PY@tok@nc\endcsname{\let\PY@bf=\textbf\def\PY@tc##1{\textcolor[rgb]{0.00,0.00,1.00}{##1}}}
\expandafter\def\csname PY@tok@nd\endcsname{\def\PY@tc##1{\textcolor[rgb]{0.67,0.13,1.00}{##1}}}
\expandafter\def\csname PY@tok@ne\endcsname{\let\PY@bf=\textbf\def\PY@tc##1{\textcolor[rgb]{0.82,0.25,0.23}{##1}}}
\expandafter\def\csname PY@tok@nf\endcsname{\def\PY@tc##1{\textcolor[rgb]{0.00,0.00,1.00}{##1}}}
\expandafter\def\csname PY@tok@si\endcsname{\let\PY@bf=\textbf\def\PY@tc##1{\textcolor[rgb]{0.73,0.40,0.53}{##1}}}
\expandafter\def\csname PY@tok@s2\endcsname{\def\PY@tc##1{\textcolor[rgb]{0.73,0.13,0.13}{##1}}}
\expandafter\def\csname PY@tok@vi\endcsname{\def\PY@tc##1{\textcolor[rgb]{0.10,0.09,0.49}{##1}}}
\expandafter\def\csname PY@tok@nt\endcsname{\let\PY@bf=\textbf\def\PY@tc##1{\textcolor[rgb]{0.00,0.50,0.00}{##1}}}
\expandafter\def\csname PY@tok@nv\endcsname{\def\PY@tc##1{\textcolor[rgb]{0.10,0.09,0.49}{##1}}}
\expandafter\def\csname PY@tok@s1\endcsname{\def\PY@tc##1{\textcolor[rgb]{0.73,0.13,0.13}{##1}}}
\expandafter\def\csname PY@tok@kd\endcsname{\let\PY@bf=\textbf\def\PY@tc##1{\textcolor[rgb]{0.00,0.50,0.00}{##1}}}
\expandafter\def\csname PY@tok@sh\endcsname{\def\PY@tc##1{\textcolor[rgb]{0.73,0.13,0.13}{##1}}}
\expandafter\def\csname PY@tok@sc\endcsname{\def\PY@tc##1{\textcolor[rgb]{0.73,0.13,0.13}{##1}}}
\expandafter\def\csname PY@tok@sx\endcsname{\def\PY@tc##1{\textcolor[rgb]{0.00,0.50,0.00}{##1}}}
\expandafter\def\csname PY@tok@bp\endcsname{\def\PY@tc##1{\textcolor[rgb]{0.00,0.50,0.00}{##1}}}
\expandafter\def\csname PY@tok@c1\endcsname{\let\PY@it=\textit\def\PY@tc##1{\textcolor[rgb]{0.25,0.50,0.50}{##1}}}
\expandafter\def\csname PY@tok@kc\endcsname{\let\PY@bf=\textbf\def\PY@tc##1{\textcolor[rgb]{0.00,0.50,0.00}{##1}}}
\expandafter\def\csname PY@tok@c\endcsname{\let\PY@it=\textit\def\PY@tc##1{\textcolor[rgb]{0.25,0.50,0.50}{##1}}}
\expandafter\def\csname PY@tok@mf\endcsname{\def\PY@tc##1{\textcolor[rgb]{0.40,0.40,0.40}{##1}}}
\expandafter\def\csname PY@tok@err\endcsname{\def\PY@bc##1{\setlength{\fboxsep}{0pt}\fcolorbox[rgb]{1.00,0.00,0.00}{1,1,1}{\strut ##1}}}
\expandafter\def\csname PY@tok@mb\endcsname{\def\PY@tc##1{\textcolor[rgb]{0.40,0.40,0.40}{##1}}}
\expandafter\def\csname PY@tok@ss\endcsname{\def\PY@tc##1{\textcolor[rgb]{0.10,0.09,0.49}{##1}}}
\expandafter\def\csname PY@tok@sr\endcsname{\def\PY@tc##1{\textcolor[rgb]{0.73,0.40,0.53}{##1}}}
\expandafter\def\csname PY@tok@mo\endcsname{\def\PY@tc##1{\textcolor[rgb]{0.40,0.40,0.40}{##1}}}
\expandafter\def\csname PY@tok@kn\endcsname{\let\PY@bf=\textbf\def\PY@tc##1{\textcolor[rgb]{0.00,0.50,0.00}{##1}}}
\expandafter\def\csname PY@tok@mi\endcsname{\def\PY@tc##1{\textcolor[rgb]{0.40,0.40,0.40}{##1}}}
\expandafter\def\csname PY@tok@gp\endcsname{\let\PY@bf=\textbf\def\PY@tc##1{\textcolor[rgb]{0.00,0.00,0.50}{##1}}}
\expandafter\def\csname PY@tok@o\endcsname{\def\PY@tc##1{\textcolor[rgb]{0.40,0.40,0.40}{##1}}}
\expandafter\def\csname PY@tok@kr\endcsname{\let\PY@bf=\textbf\def\PY@tc##1{\textcolor[rgb]{0.00,0.50,0.00}{##1}}}
\expandafter\def\csname PY@tok@s\endcsname{\def\PY@tc##1{\textcolor[rgb]{0.73,0.13,0.13}{##1}}}
\expandafter\def\csname PY@tok@kp\endcsname{\def\PY@tc##1{\textcolor[rgb]{0.00,0.50,0.00}{##1}}}
\expandafter\def\csname PY@tok@w\endcsname{\def\PY@tc##1{\textcolor[rgb]{0.73,0.73,0.73}{##1}}}
\expandafter\def\csname PY@tok@kt\endcsname{\def\PY@tc##1{\textcolor[rgb]{0.69,0.00,0.25}{##1}}}
\expandafter\def\csname PY@tok@ow\endcsname{\let\PY@bf=\textbf\def\PY@tc##1{\textcolor[rgb]{0.67,0.13,1.00}{##1}}}
\expandafter\def\csname PY@tok@sb\endcsname{\def\PY@tc##1{\textcolor[rgb]{0.73,0.13,0.13}{##1}}}
\expandafter\def\csname PY@tok@k\endcsname{\let\PY@bf=\textbf\def\PY@tc##1{\textcolor[rgb]{0.00,0.50,0.00}{##1}}}
\expandafter\def\csname PY@tok@se\endcsname{\let\PY@bf=\textbf\def\PY@tc##1{\textcolor[rgb]{0.73,0.40,0.13}{##1}}}
\expandafter\def\csname PY@tok@sd\endcsname{\let\PY@it=\textit\def\PY@tc##1{\textcolor[rgb]{0.73,0.13,0.13}{##1}}}

\def\PYZbs{\char`\\}
\def\PYZus{\char`\_}
\def\PYZob{\char`\{}
\def\PYZcb{\char`\}}
\def\PYZca{\char`\^}
\def\PYZam{\char`\&}
\def\PYZlt{\char`\<}
\def\PYZgt{\char`\>}
\def\PYZsh{\char`\#}
\def\PYZpc{\char`\%}
\def\PYZdl{\char`\$}
\def\PYZhy{\char`\-}
\def\PYZsq{\char`\'}
\def\PYZdq{\char`\"}
\def\PYZti{\char`\~}
% for compatibility with earlier versions
\def\PYZat{@}
\def\PYZlb{[}
\def\PYZrb{]}
\makeatother


    % Exact colors from NB
    \definecolor{incolor}{rgb}{0.0, 0.0, 0.5}
    \definecolor{outcolor}{rgb}{0.545, 0.0, 0.0}



    
    % Prevent overflowing lines due to hard-to-break entities
    \sloppy 
    % Setup hyperref package
    \hypersetup{
      breaklinks=true,  % so long urls are correctly broken across lines
      colorlinks=true,
      urlcolor=blue,
      linkcolor=darkorange,
      citecolor=darkgreen,
      }
    % Slightly bigger margins than the latex defaults
    
    \geometry{verbose,tmargin=1in,bmargin=1in,lmargin=1in,rmargin=1in}
    
    

    \begin{document}
    
\begin{titlepage}
	\title{\Huge{Get Started with Python}}
	\author{Rajath Kumar M.P. \\ \\  Department of Electronics and Communication Engineering \\ RNS Institute of Technology, Bangalore \\ \\rajathkumar.exe@gmail.com}
	\date{}
	\maketitle
	\thispagestyle{empty}
\end{titlepage}


   
    
    
 \tableofcontents   
 
 \newpage
    

    
    \section{Python}\label{python}

    \subsection{Introduction}\label{introduction}

    Python is a modern, robust, high level programming language. It is very
easy to pick up even if you are completely new to programming.

    \subsection{Installation}\label{installation}

    Mac OS X and Linux comes pre installed with python. Windows users can
download python from https://www.python.org/downloads/ .

To install IPython run,

\begin{verbatim}
$ pip install ipython[all]
\end{verbatim}

This will install all the necessary dependencies for the notebook,
qtconsole, tests etc.

    \subsubsection{Installation from unofficial
distributions}\label{installation-from-unofficial-distributions}

    Installing all the necessary libraries might prove troublesome. Anaconda
and Canopy comes pre packaged with all the necessary python libraries
and also IPython.

    \paragraph{Anaconda}\label{anaconda}

    Download Anaconda from https://www.continuum.io/downloads

Anaconda is completely free and includes more than 300 python packages.
Both python 2.7 and 3.4 options are available.

    \paragraph{Canopy}\label{canopy}

    Download Canopy from https://store.enthought.com/downloads/\#default

Canopy has a premium version which offers 300+ python packages. But the
free version works just fine. Canopy as of now supports only 2.7 but it
comes with its own text editor and IPython environment.

    \subsection{Launching IPython
Notebook}\label{launching-ipython-notebook}

    From the terminal

\begin{verbatim}
ipython notebook
\end{verbatim}

In Canopy and Anaconda, Open the respective terminals and execute the
above.

    \subsection{How to learn from this
resource?}\label{how-to-learn-from-this-resource}

    Download all the ipython notebooks from this repository
https://github.com/rajathkumarmp/Python-Lectures

Launch ipython notebook from the folder which contains the notebooks.
Open each one of them

\begin{verbatim}
Cell > All Output > Clear
\end{verbatim}

This will clear all the outputs and now you can understand each
statement and learn interactively.

    \subsection{License}\label{license}

    This work is licensed under the Creative Commons Attribution 3.0
Unported License. To view a copy of this license, visit
http://creativecommons.org/licenses/by/3.0/


    % Add a bibliography block to the postdoc
    
 \newpage
 
 
% Default to the notebook output style

    


% Inherit from the specified cell style.




    


    
 
    
    \definecolor{orange}{cmyk}{0,0.4,0.8,0.2}
    \definecolor{darkorange}{rgb}{.71,0.21,0.01}
    \definecolor{darkgreen}{rgb}{.12,.54,.11}
    \definecolor{myteal}{rgb}{.26, .44, .56}
    \definecolor{gray}{gray}{0.45}
    \definecolor{lightgray}{gray}{.95}
    \definecolor{mediumgray}{gray}{.8}
    \definecolor{inputbackground}{rgb}{.95, .95, .85}
    \definecolor{outputbackground}{rgb}{.95, .95, .95}
    \definecolor{traceback}{rgb}{1, .95, .95}
    % ansi colors
    \definecolor{red}{rgb}{.6,0,0}
    \definecolor{green}{rgb}{0,.65,0}
    \definecolor{brown}{rgb}{0.6,0.6,0}
    \definecolor{blue}{rgb}{0,.145,.698}
    \definecolor{purple}{rgb}{.698,.145,.698}
    \definecolor{cyan}{rgb}{0,.698,.698}
    \definecolor{lightgray}{gray}{0.5}
    
    % bright ansi colors
    \definecolor{darkgray}{gray}{0.25}
    \definecolor{lightred}{rgb}{1.0,0.39,0.28}
    \definecolor{lightgreen}{rgb}{0.48,0.99,0.0}
    \definecolor{lightblue}{rgb}{0.53,0.81,0.92}
    \definecolor{lightpurple}{rgb}{0.87,0.63,0.87}
    \definecolor{lightcyan}{rgb}{0.5,1.0,0.83}
    
    % commands and environments needed by pandoc snippets
    % extracted from the output of `pandoc -s`
    \DefineVerbatimEnvironment{Highlighting}{Verbatim}{commandchars=\\\{\}}
    % Add ',fontsize=\small' for more characters per line
    \newenvironment{Shaded}{}{}
    \newcommand{\KeywordTok}[1]{\textcolor[rgb]{0.00,0.44,0.13}{\textbf{{#1}}}}
    \newcommand{\DataTypeTok}[1]{\textcolor[rgb]{0.56,0.13,0.00}{{#1}}}
    \newcommand{\DecValTok}[1]{\textcolor[rgb]{0.25,0.63,0.44}{{#1}}}
    \newcommand{\BaseNTok}[1]{\textcolor[rgb]{0.25,0.63,0.44}{{#1}}}
    \newcommand{\FloatTok}[1]{\textcolor[rgb]{0.25,0.63,0.44}{{#1}}}
    \newcommand{\CharTok}[1]{\textcolor[rgb]{0.25,0.44,0.63}{{#1}}}
    \newcommand{\StringTok}[1]{\textcolor[rgb]{0.25,0.44,0.63}{{#1}}}
    \newcommand{\CommentTok}[1]{\textcolor[rgb]{0.38,0.63,0.69}{\textit{{#1}}}}
    \newcommand{\OtherTok}[1]{\textcolor[rgb]{0.00,0.44,0.13}{{#1}}}
    \newcommand{\AlertTok}[1]{\textcolor[rgb]{1.00,0.00,0.00}{\textbf{{#1}}}}
    \newcommand{\FunctionTok}[1]{\textcolor[rgb]{0.02,0.16,0.49}{{#1}}}
    \newcommand{\RegionMarkerTok}[1]{{#1}}
    \newcommand{\ErrorTok}[1]{\textcolor[rgb]{1.00,0.00,0.00}{\textbf{{#1}}}}
    \newcommand{\NormalTok}[1]{{#1}}
    
    % Define a nice break command that doesn't care if a line doesn't already
    % exist.
    \def\br{\hspace*{\fill} \\* }
    % Math Jax compatability definitions
    \def\gt{>}
    \def\lt{<}
    % Document parameters
    \title{}
    
    
    

    % Pygments definitions
    
\makeatletter
\def\PY@reset{\let\PY@it=\relax \let\PY@bf=\relax%
    \let\PY@ul=\relax \let\PY@tc=\relax%
    \let\PY@bc=\relax \let\PY@ff=\relax}
\def\PY@tok#1{\csname PY@tok@#1\endcsname}
\def\PY@toks#1+{\ifx\relax#1\empty\else%
    \PY@tok{#1}\expandafter\PY@toks\fi}
\def\PY@do#1{\PY@bc{\PY@tc{\PY@ul{%
    \PY@it{\PY@bf{\PY@ff{#1}}}}}}}
\def\PY#1#2{\PY@reset\PY@toks#1+\relax+\PY@do{#2}}

\expandafter\def\csname PY@tok@gd\endcsname{\def\PY@tc##1{\textcolor[rgb]{0.63,0.00,0.00}{##1}}}
\expandafter\def\csname PY@tok@gu\endcsname{\let\PY@bf=\textbf\def\PY@tc##1{\textcolor[rgb]{0.50,0.00,0.50}{##1}}}
\expandafter\def\csname PY@tok@gt\endcsname{\def\PY@tc##1{\textcolor[rgb]{0.00,0.27,0.87}{##1}}}
\expandafter\def\csname PY@tok@gs\endcsname{\let\PY@bf=\textbf}
\expandafter\def\csname PY@tok@gr\endcsname{\def\PY@tc##1{\textcolor[rgb]{1.00,0.00,0.00}{##1}}}
\expandafter\def\csname PY@tok@cm\endcsname{\let\PY@it=\textit\def\PY@tc##1{\textcolor[rgb]{0.25,0.50,0.50}{##1}}}
\expandafter\def\csname PY@tok@vg\endcsname{\def\PY@tc##1{\textcolor[rgb]{0.10,0.09,0.49}{##1}}}
\expandafter\def\csname PY@tok@m\endcsname{\def\PY@tc##1{\textcolor[rgb]{0.40,0.40,0.40}{##1}}}
\expandafter\def\csname PY@tok@mh\endcsname{\def\PY@tc##1{\textcolor[rgb]{0.40,0.40,0.40}{##1}}}
\expandafter\def\csname PY@tok@go\endcsname{\def\PY@tc##1{\textcolor[rgb]{0.53,0.53,0.53}{##1}}}
\expandafter\def\csname PY@tok@ge\endcsname{\let\PY@it=\textit}
\expandafter\def\csname PY@tok@vc\endcsname{\def\PY@tc##1{\textcolor[rgb]{0.10,0.09,0.49}{##1}}}
\expandafter\def\csname PY@tok@il\endcsname{\def\PY@tc##1{\textcolor[rgb]{0.40,0.40,0.40}{##1}}}
\expandafter\def\csname PY@tok@cs\endcsname{\let\PY@it=\textit\def\PY@tc##1{\textcolor[rgb]{0.25,0.50,0.50}{##1}}}
\expandafter\def\csname PY@tok@cp\endcsname{\def\PY@tc##1{\textcolor[rgb]{0.74,0.48,0.00}{##1}}}
\expandafter\def\csname PY@tok@gi\endcsname{\def\PY@tc##1{\textcolor[rgb]{0.00,0.63,0.00}{##1}}}
\expandafter\def\csname PY@tok@gh\endcsname{\let\PY@bf=\textbf\def\PY@tc##1{\textcolor[rgb]{0.00,0.00,0.50}{##1}}}
\expandafter\def\csname PY@tok@ni\endcsname{\let\PY@bf=\textbf\def\PY@tc##1{\textcolor[rgb]{0.60,0.60,0.60}{##1}}}
\expandafter\def\csname PY@tok@nl\endcsname{\def\PY@tc##1{\textcolor[rgb]{0.63,0.63,0.00}{##1}}}
\expandafter\def\csname PY@tok@nn\endcsname{\let\PY@bf=\textbf\def\PY@tc##1{\textcolor[rgb]{0.00,0.00,1.00}{##1}}}
\expandafter\def\csname PY@tok@no\endcsname{\def\PY@tc##1{\textcolor[rgb]{0.53,0.00,0.00}{##1}}}
\expandafter\def\csname PY@tok@na\endcsname{\def\PY@tc##1{\textcolor[rgb]{0.49,0.56,0.16}{##1}}}
\expandafter\def\csname PY@tok@nb\endcsname{\def\PY@tc##1{\textcolor[rgb]{0.00,0.50,0.00}{##1}}}
\expandafter\def\csname PY@tok@nc\endcsname{\let\PY@bf=\textbf\def\PY@tc##1{\textcolor[rgb]{0.00,0.00,1.00}{##1}}}
\expandafter\def\csname PY@tok@nd\endcsname{\def\PY@tc##1{\textcolor[rgb]{0.67,0.13,1.00}{##1}}}
\expandafter\def\csname PY@tok@ne\endcsname{\let\PY@bf=\textbf\def\PY@tc##1{\textcolor[rgb]{0.82,0.25,0.23}{##1}}}
\expandafter\def\csname PY@tok@nf\endcsname{\def\PY@tc##1{\textcolor[rgb]{0.00,0.00,1.00}{##1}}}
\expandafter\def\csname PY@tok@si\endcsname{\let\PY@bf=\textbf\def\PY@tc##1{\textcolor[rgb]{0.73,0.40,0.53}{##1}}}
\expandafter\def\csname PY@tok@s2\endcsname{\def\PY@tc##1{\textcolor[rgb]{0.73,0.13,0.13}{##1}}}
\expandafter\def\csname PY@tok@vi\endcsname{\def\PY@tc##1{\textcolor[rgb]{0.10,0.09,0.49}{##1}}}
\expandafter\def\csname PY@tok@nt\endcsname{\let\PY@bf=\textbf\def\PY@tc##1{\textcolor[rgb]{0.00,0.50,0.00}{##1}}}
\expandafter\def\csname PY@tok@nv\endcsname{\def\PY@tc##1{\textcolor[rgb]{0.10,0.09,0.49}{##1}}}
\expandafter\def\csname PY@tok@s1\endcsname{\def\PY@tc##1{\textcolor[rgb]{0.73,0.13,0.13}{##1}}}
\expandafter\def\csname PY@tok@kd\endcsname{\let\PY@bf=\textbf\def\PY@tc##1{\textcolor[rgb]{0.00,0.50,0.00}{##1}}}
\expandafter\def\csname PY@tok@sh\endcsname{\def\PY@tc##1{\textcolor[rgb]{0.73,0.13,0.13}{##1}}}
\expandafter\def\csname PY@tok@sc\endcsname{\def\PY@tc##1{\textcolor[rgb]{0.73,0.13,0.13}{##1}}}
\expandafter\def\csname PY@tok@sx\endcsname{\def\PY@tc##1{\textcolor[rgb]{0.00,0.50,0.00}{##1}}}
\expandafter\def\csname PY@tok@bp\endcsname{\def\PY@tc##1{\textcolor[rgb]{0.00,0.50,0.00}{##1}}}
\expandafter\def\csname PY@tok@c1\endcsname{\let\PY@it=\textit\def\PY@tc##1{\textcolor[rgb]{0.25,0.50,0.50}{##1}}}
\expandafter\def\csname PY@tok@kc\endcsname{\let\PY@bf=\textbf\def\PY@tc##1{\textcolor[rgb]{0.00,0.50,0.00}{##1}}}
\expandafter\def\csname PY@tok@c\endcsname{\let\PY@it=\textit\def\PY@tc##1{\textcolor[rgb]{0.25,0.50,0.50}{##1}}}
\expandafter\def\csname PY@tok@mf\endcsname{\def\PY@tc##1{\textcolor[rgb]{0.40,0.40,0.40}{##1}}}
\expandafter\def\csname PY@tok@err\endcsname{\def\PY@bc##1{\setlength{\fboxsep}{0pt}\fcolorbox[rgb]{1.00,0.00,0.00}{1,1,1}{\strut ##1}}}
\expandafter\def\csname PY@tok@mb\endcsname{\def\PY@tc##1{\textcolor[rgb]{0.40,0.40,0.40}{##1}}}
\expandafter\def\csname PY@tok@ss\endcsname{\def\PY@tc##1{\textcolor[rgb]{0.10,0.09,0.49}{##1}}}
\expandafter\def\csname PY@tok@sr\endcsname{\def\PY@tc##1{\textcolor[rgb]{0.73,0.40,0.53}{##1}}}
\expandafter\def\csname PY@tok@mo\endcsname{\def\PY@tc##1{\textcolor[rgb]{0.40,0.40,0.40}{##1}}}
\expandafter\def\csname PY@tok@kn\endcsname{\let\PY@bf=\textbf\def\PY@tc##1{\textcolor[rgb]{0.00,0.50,0.00}{##1}}}
\expandafter\def\csname PY@tok@mi\endcsname{\def\PY@tc##1{\textcolor[rgb]{0.40,0.40,0.40}{##1}}}
\expandafter\def\csname PY@tok@gp\endcsname{\let\PY@bf=\textbf\def\PY@tc##1{\textcolor[rgb]{0.00,0.00,0.50}{##1}}}
\expandafter\def\csname PY@tok@o\endcsname{\def\PY@tc##1{\textcolor[rgb]{0.40,0.40,0.40}{##1}}}
\expandafter\def\csname PY@tok@kr\endcsname{\let\PY@bf=\textbf\def\PY@tc##1{\textcolor[rgb]{0.00,0.50,0.00}{##1}}}
\expandafter\def\csname PY@tok@s\endcsname{\def\PY@tc##1{\textcolor[rgb]{0.73,0.13,0.13}{##1}}}
\expandafter\def\csname PY@tok@kp\endcsname{\def\PY@tc##1{\textcolor[rgb]{0.00,0.50,0.00}{##1}}}
\expandafter\def\csname PY@tok@w\endcsname{\def\PY@tc##1{\textcolor[rgb]{0.73,0.73,0.73}{##1}}}
\expandafter\def\csname PY@tok@kt\endcsname{\def\PY@tc##1{\textcolor[rgb]{0.69,0.00,0.25}{##1}}}
\expandafter\def\csname PY@tok@ow\endcsname{\let\PY@bf=\textbf\def\PY@tc##1{\textcolor[rgb]{0.67,0.13,1.00}{##1}}}
\expandafter\def\csname PY@tok@sb\endcsname{\def\PY@tc##1{\textcolor[rgb]{0.73,0.13,0.13}{##1}}}
\expandafter\def\csname PY@tok@k\endcsname{\let\PY@bf=\textbf\def\PY@tc##1{\textcolor[rgb]{0.00,0.50,0.00}{##1}}}
\expandafter\def\csname PY@tok@se\endcsname{\let\PY@bf=\textbf\def\PY@tc##1{\textcolor[rgb]{0.73,0.40,0.13}{##1}}}
\expandafter\def\csname PY@tok@sd\endcsname{\let\PY@it=\textit\def\PY@tc##1{\textcolor[rgb]{0.73,0.13,0.13}{##1}}}

\def\PYZbs{\char`\\}
\def\PYZus{\char`\_}
\def\PYZob{\char`\{}
\def\PYZcb{\char`\}}
\def\PYZca{\char`\^}
\def\PYZam{\char`\&}
\def\PYZlt{\char`\<}
\def\PYZgt{\char`\>}
\def\PYZsh{\char`\#}
\def\PYZpc{\char`\%}
\def\PYZdl{\char`\$}
\def\PYZhy{\char`\-}
\def\PYZsq{\char`\'}
\def\PYZdq{\char`\"}
\def\PYZti{\char`\~}
% for compatibility with earlier versions
\def\PYZat{@}
\def\PYZlb{[}
\def\PYZrb{]}
\makeatother


    % Exact colors from NB
    \definecolor{incolor}{rgb}{0.0, 0.0, 0.5}
    \definecolor{outcolor}{rgb}{0.545, 0.0, 0.0}



    
    % Prevent overflowing lines due to hard-to-break entities
    \sloppy 
    % Setup hyperref package
    \hypersetup{
      breaklinks=true,  % so long urls are correctly broken across lines
      colorlinks=true,
      urlcolor=blue,
      linkcolor=darkorange,
      citecolor=darkgreen,
      }
    % Slightly bigger margins than the latex defaults
    
    
    

    \begin{document}
    
    
    \maketitle
    
    

    
    \section{The Zen Of Python}\label{the-zen-of-python}

    \begin{Verbatim}[commandchars=\\\{\}]
{\color{incolor}In [{\color{incolor}1}]:} \PY{k+kn}{import} \PY{n+nn}{this}
\end{Verbatim}

    \begin{Verbatim}[commandchars=\\\{\}]
The Zen of Python, by Tim Peters

Beautiful is better than ugly.
Explicit is better than implicit.
Simple is better than complex.
Complex is better than complicated.
Flat is better than nested.
Sparse is better than dense.
Readability counts.
Special cases aren't special enough to break the rules.
Although practicality beats purity.
Errors should never pass silently.
Unless explicitly silenced.
In the face of ambiguity, refuse the temptation to guess.
There should be one-- and preferably only one --obvious way to do it.
Although that way may not be obvious at first unless you're Dutch.
Now is better than never.
Although never is often better than *right* now.
If the implementation is hard to explain, it's a bad idea.
If the implementation is easy to explain, it may be a good idea.
Namespaces are one honking great idea -- let's do more of those!
    \end{Verbatim}

    \section{Variables}\label{variables}

    A name that is used to denote something or a value is called a variable.
In python, variables can be declared and values can be assigned to it as
follows,

    \begin{Verbatim}[commandchars=\\\{\}]
{\color{incolor}In [{\color{incolor}2}]:} \PY{n}{x} \PY{o}{=} \PY{l+m+mi}{2}
        \PY{n}{y} \PY{o}{=} \PY{l+m+mi}{5}
        \PY{n}{xy} \PY{o}{=} \PY{l+s}{\PYZsq{}}\PY{l+s}{Hey}\PY{l+s}{\PYZsq{}}
\end{Verbatim}

    \begin{Verbatim}[commandchars=\\\{\}]
{\color{incolor}In [{\color{incolor}3}]:} \PY{k}{print} \PY{n}{x}\PY{o}{+}\PY{n}{y}\PY{p}{,} \PY{n}{xy}
\end{Verbatim}

    \begin{Verbatim}[commandchars=\\\{\}]
7 Hey
    \end{Verbatim}

    Multiple variables can be assigned with the same value.

    \begin{Verbatim}[commandchars=\\\{\}]
{\color{incolor}In [{\color{incolor}4}]:} \PY{n}{x} \PY{o}{=} \PY{n}{y} \PY{o}{=} \PY{l+m+mi}{1}
\end{Verbatim}

    \begin{Verbatim}[commandchars=\\\{\}]
{\color{incolor}In [{\color{incolor}5}]:} \PY{k}{print} \PY{n}{x}\PY{p}{,}\PY{n}{y}
\end{Verbatim}

    \begin{Verbatim}[commandchars=\\\{\}]
1 1
    \end{Verbatim}

    \section{Operators}\label{operators}

    \subsection{Arithmetic Operators}\label{arithmetic-operators}

    \begin{longtable}[c]{@{}ll@{}}
\toprule
Symbol & Task Performed\tabularnewline
\midrule
\endhead
+ & Addition\tabularnewline
- & Subtraction\tabularnewline
/ & division\tabularnewline
\% & mod\tabularnewline
* & multiplication\tabularnewline
// & floor division\tabularnewline
** & to the power of\tabularnewline
\bottomrule
\end{longtable}

    \begin{Verbatim}[commandchars=\\\{\}]
{\color{incolor}In [{\color{incolor}6}]:} \PY{l+m+mi}{1}\PY{o}{+}\PY{l+m+mi}{2}
\end{Verbatim}

            \begin{Verbatim}[commandchars=\\\{\}]
{\color{outcolor}Out[{\color{outcolor}6}]:} 3
\end{Verbatim}
        
    \begin{Verbatim}[commandchars=\\\{\}]
{\color{incolor}In [{\color{incolor}7}]:} \PY{l+m+mi}{2}\PY{o}{\PYZhy{}}\PY{l+m+mi}{1}
\end{Verbatim}

            \begin{Verbatim}[commandchars=\\\{\}]
{\color{outcolor}Out[{\color{outcolor}7}]:} 1
\end{Verbatim}
        
    \begin{Verbatim}[commandchars=\\\{\}]
{\color{incolor}In [{\color{incolor}8}]:} \PY{l+m+mi}{1}\PY{o}{*}\PY{l+m+mi}{2}
\end{Verbatim}

            \begin{Verbatim}[commandchars=\\\{\}]
{\color{outcolor}Out[{\color{outcolor}8}]:} 2
\end{Verbatim}
        
    \begin{Verbatim}[commandchars=\\\{\}]
{\color{incolor}In [{\color{incolor}9}]:} \PY{l+m+mi}{1}\PY{o}{/}\PY{l+m+mi}{2}
\end{Verbatim}

            \begin{Verbatim}[commandchars=\\\{\}]
{\color{outcolor}Out[{\color{outcolor}9}]:} 0
\end{Verbatim}
        
    0? This is because both the numerator and denominator are integers but
the result is a float value hence an integer value is returned. By
changing either the numerator or the denominator to float, correct
answer can be obtained.

    \begin{Verbatim}[commandchars=\\\{\}]
{\color{incolor}In [{\color{incolor}10}]:} \PY{l+m+mi}{1}\PY{o}{/}\PY{l+m+mf}{2.0}
\end{Verbatim}

            \begin{Verbatim}[commandchars=\\\{\}]
{\color{outcolor}Out[{\color{outcolor}10}]:} 0.5
\end{Verbatim}
        
    \begin{Verbatim}[commandchars=\\\{\}]
{\color{incolor}In [{\color{incolor}11}]:} \PY{l+m+mi}{15}\PY{o}{\PYZpc{}}\PY{k}{10}
\end{Verbatim}

            \begin{Verbatim}[commandchars=\\\{\}]
{\color{outcolor}Out[{\color{outcolor}11}]:} 5
\end{Verbatim}
        
    Floor division is nothing but converting the result so obtained to the
nearest integer.

    \begin{Verbatim}[commandchars=\\\{\}]
{\color{incolor}In [{\color{incolor}12}]:} \PY{l+m+mf}{2.8}\PY{o}{/}\PY{o}{/}\PY{l+m+mf}{2.0}
\end{Verbatim}

            \begin{Verbatim}[commandchars=\\\{\}]
{\color{outcolor}Out[{\color{outcolor}12}]:} 1.0
\end{Verbatim}
        
    \subsection{Relational Operators}\label{relational-operators}

    \begin{longtable}[c]{@{}ll@{}}
\toprule
Symbol & Task Performed\tabularnewline
\midrule
\endhead
== & True, if it is equal\tabularnewline
!= & True, if not equal to\tabularnewline
\textless{} & less than\tabularnewline
\textgreater{} & greater than\tabularnewline
\textless{}= & less than or equal to\tabularnewline
\textgreater{}= & greater than or equal to\tabularnewline
\bottomrule
\end{longtable}

    \begin{Verbatim}[commandchars=\\\{\}]
{\color{incolor}In [{\color{incolor}13}]:} \PY{n}{z} \PY{o}{=} \PY{l+m+mi}{1}
\end{Verbatim}

    \begin{Verbatim}[commandchars=\\\{\}]
{\color{incolor}In [{\color{incolor}14}]:} \PY{n}{z} \PY{o}{==} \PY{l+m+mi}{1}
\end{Verbatim}

            \begin{Verbatim}[commandchars=\\\{\}]
{\color{outcolor}Out[{\color{outcolor}14}]:} True
\end{Verbatim}
        
    \begin{Verbatim}[commandchars=\\\{\}]
{\color{incolor}In [{\color{incolor}15}]:} \PY{n}{z} \PY{o}{\PYZgt{}} \PY{l+m+mi}{1}
\end{Verbatim}

            \begin{Verbatim}[commandchars=\\\{\}]
{\color{outcolor}Out[{\color{outcolor}15}]:} False
\end{Verbatim}
        
    \subsection{Bitwise Operators}\label{bitwise-operators}

    \begin{longtable}[c]{@{}ll@{}}
\toprule
Symbol & Task Performed\tabularnewline
\midrule
\endhead
\& & Logical And\tabularnewline
l & Logical OR\tabularnewline
\^{} & XOR\tabularnewline
\textasciitilde{} & Negate\tabularnewline
\textgreater{}\textgreater{} & Right shift\tabularnewline
\textless{}\textless{} & Left shift\tabularnewline
\bottomrule
\end{longtable}

    \begin{Verbatim}[commandchars=\\\{\}]
{\color{incolor}In [{\color{incolor}16}]:} \PY{n}{a} \PY{o}{=} \PY{l+m+mi}{2} \PY{c}{\PYZsh{}10}
         \PY{n}{b} \PY{o}{=} \PY{l+m+mi}{3} \PY{c}{\PYZsh{}11}
\end{Verbatim}

    \begin{Verbatim}[commandchars=\\\{\}]
{\color{incolor}In [{\color{incolor}17}]:} \PY{k}{print} \PY{n}{a} \PY{o}{\PYZam{}} \PY{n}{b}
         \PY{k}{print} \PY{n+nb}{bin}\PY{p}{(}\PY{n}{a}\PY{o}{\PYZam{}}\PY{n}{b}\PY{p}{)}
\end{Verbatim}

    \begin{Verbatim}[commandchars=\\\{\}]
2
0b10
    \end{Verbatim}

    \begin{Verbatim}[commandchars=\\\{\}]
{\color{incolor}In [{\color{incolor}18}]:} \PY{l+m+mi}{5} \PY{o}{\PYZgt{}\PYZgt{}} \PY{l+m+mi}{1}
\end{Verbatim}

            \begin{Verbatim}[commandchars=\\\{\}]
{\color{outcolor}Out[{\color{outcolor}18}]:} 2
\end{Verbatim}
        
    0000 0101 -\textgreater{} 5

Shifting the digits by 1 to the right and zero padding

0000 0010 -\textgreater{} 2

    \begin{Verbatim}[commandchars=\\\{\}]
{\color{incolor}In [{\color{incolor}19}]:} \PY{l+m+mi}{5} \PY{o}{\PYZlt{}\PYZlt{}} \PY{l+m+mi}{1}
\end{Verbatim}

            \begin{Verbatim}[commandchars=\\\{\}]
{\color{outcolor}Out[{\color{outcolor}19}]:} 10
\end{Verbatim}
        
    0000 0101 -\textgreater{} 5

Shifting the digits by 1 to the left and zero padding

0000 1010 -\textgreater{} 10

    \section{Built-in Functions}\label{built-in-functions}

    Python comes loaded with pre-built functions

    \subsection{Conversion from one system to
another}\label{conversion-from-one-system-to-another}

    Conversion from hexadecimal to decimal is done by adding prefix
\textbf{0x} to the hexadecimal value or vice versa by using built in
\textbf{hex( )}, Octal to decimal by adding prefix \textbf{0} to the
octal value or vice versa by using built in function \textbf{oct( )}.

    \begin{Verbatim}[commandchars=\\\{\}]
{\color{incolor}In [{\color{incolor}20}]:} \PY{n+nb}{hex}\PY{p}{(}\PY{l+m+mi}{170}\PY{p}{)}
\end{Verbatim}

            \begin{Verbatim}[commandchars=\\\{\}]
{\color{outcolor}Out[{\color{outcolor}20}]:} '0xaa'
\end{Verbatim}
        
    \begin{Verbatim}[commandchars=\\\{\}]
{\color{incolor}In [{\color{incolor}21}]:} \PY{l+m+mh}{0xAA}
\end{Verbatim}

            \begin{Verbatim}[commandchars=\\\{\}]
{\color{outcolor}Out[{\color{outcolor}21}]:} 170
\end{Verbatim}
        
    \begin{Verbatim}[commandchars=\\\{\}]
{\color{incolor}In [{\color{incolor}22}]:} \PY{n+nb}{oct}\PY{p}{(}\PY{l+m+mi}{8}\PY{p}{)}
\end{Verbatim}

            \begin{Verbatim}[commandchars=\\\{\}]
{\color{outcolor}Out[{\color{outcolor}22}]:} '010'
\end{Verbatim}
        
    \begin{Verbatim}[commandchars=\\\{\}]
{\color{incolor}In [{\color{incolor}23}]:} \PY{l+m+mo}{010}
\end{Verbatim}

            \begin{Verbatim}[commandchars=\\\{\}]
{\color{outcolor}Out[{\color{outcolor}23}]:} 8
\end{Verbatim}
        
    \textbf{int( )} accepts two values when used for conversion, one is the
value in a different number system and the other is its base. Note that
input number in the different number system should be of string type.

    \begin{Verbatim}[commandchars=\\\{\}]
{\color{incolor}In [{\color{incolor}24}]:} \PY{k}{print} \PY{n+nb}{int}\PY{p}{(}\PY{l+s}{\PYZsq{}}\PY{l+s}{010}\PY{l+s}{\PYZsq{}}\PY{p}{,}\PY{l+m+mi}{8}\PY{p}{)}
         \PY{k}{print} \PY{n+nb}{int}\PY{p}{(}\PY{l+s}{\PYZsq{}}\PY{l+s}{0xaa}\PY{l+s}{\PYZsq{}}\PY{p}{,}\PY{l+m+mi}{16}\PY{p}{)}
         \PY{k}{print} \PY{n+nb}{int}\PY{p}{(}\PY{l+s}{\PYZsq{}}\PY{l+s}{1010}\PY{l+s}{\PYZsq{}}\PY{p}{,}\PY{l+m+mi}{2}\PY{p}{)}
\end{Verbatim}

    \begin{Verbatim}[commandchars=\\\{\}]
8
170
10
    \end{Verbatim}

    \textbf{int( )} can also be used to get only the integer value of a
float number or can be used to convert a number which is of type string
to integer format. Similarly, the function \textbf{str( )} can be used
to convert the integer back to string format

    \begin{Verbatim}[commandchars=\\\{\}]
{\color{incolor}In [{\color{incolor}25}]:} \PY{k}{print} \PY{n+nb}{int}\PY{p}{(}\PY{l+m+mf}{7.7}\PY{p}{)}
         \PY{k}{print} \PY{n+nb}{int}\PY{p}{(}\PY{l+s}{\PYZsq{}}\PY{l+s}{7}\PY{l+s}{\PYZsq{}}\PY{p}{)}
\end{Verbatim}

    \begin{Verbatim}[commandchars=\\\{\}]
7
7
    \end{Verbatim}

    Also note that function \textbf{bin( )} is used for binary and
\textbf{float( )} for decimal/float values. \textbf{chr( )} is used for
converting ASCII to its alphabet equivalent, \textbf{ord( )} is used for
the other way round.

    \begin{Verbatim}[commandchars=\\\{\}]
{\color{incolor}In [{\color{incolor}26}]:} \PY{n+nb}{chr}\PY{p}{(}\PY{l+m+mi}{98}\PY{p}{)}
\end{Verbatim}

            \begin{Verbatim}[commandchars=\\\{\}]
{\color{outcolor}Out[{\color{outcolor}26}]:} 'b'
\end{Verbatim}
        
    \begin{Verbatim}[commandchars=\\\{\}]
{\color{incolor}In [{\color{incolor}27}]:} \PY{n+nb}{ord}\PY{p}{(}\PY{l+s}{\PYZsq{}}\PY{l+s}{b}\PY{l+s}{\PYZsq{}}\PY{p}{)}
\end{Verbatim}

            \begin{Verbatim}[commandchars=\\\{\}]
{\color{outcolor}Out[{\color{outcolor}27}]:} 98
\end{Verbatim}
        
    \subsection{Simplifying Arithmetic
Operations}\label{simplifying-arithmetic-operations}

    \textbf{round( )} function rounds the input value to a specified number
of places or to the nearest integer.

    \begin{Verbatim}[commandchars=\\\{\}]
{\color{incolor}In [{\color{incolor}28}]:} \PY{k}{print} \PY{n+nb}{round}\PY{p}{(}\PY{l+m+mf}{5.6231}\PY{p}{)} 
         \PY{k}{print} \PY{n+nb}{round}\PY{p}{(}\PY{l+m+mf}{4.55892}\PY{p}{,} \PY{l+m+mi}{2}\PY{p}{)}
\end{Verbatim}

    \begin{Verbatim}[commandchars=\\\{\}]
6.0
4.56
    \end{Verbatim}

    \textbf{complex( )} is used to define a complex number and \textbf{abs(
)} outputs the absolute value of the same.

    \begin{Verbatim}[commandchars=\\\{\}]
{\color{incolor}In [{\color{incolor}29}]:} \PY{n}{c} \PY{o}{=}\PY{n+nb}{complex}\PY{p}{(}\PY{l+s}{\PYZsq{}}\PY{l+s}{5+2j}\PY{l+s}{\PYZsq{}}\PY{p}{)}
         \PY{k}{print} \PY{n+nb}{abs}\PY{p}{(}\PY{n}{c}\PY{p}{)}
\end{Verbatim}

    \begin{Verbatim}[commandchars=\\\{\}]
5.38516480713
    \end{Verbatim}

    \textbf{divmod(x,y)} outputs the quotient and the remainder in a
tuple(you will be learning about it in the further chapters) in the
format (quotient, remainder).

    \begin{Verbatim}[commandchars=\\\{\}]
{\color{incolor}In [{\color{incolor}30}]:} \PY{n+nb}{divmod}\PY{p}{(}\PY{l+m+mi}{9}\PY{p}{,}\PY{l+m+mi}{2}\PY{p}{)}
\end{Verbatim}

            \begin{Verbatim}[commandchars=\\\{\}]
{\color{outcolor}Out[{\color{outcolor}30}]:} (4, 1)
\end{Verbatim}
        
    \textbf{isinstance( )} returns True, if the first argument is an
instance of that class. Multiple classes can also be checked at once.

    \begin{Verbatim}[commandchars=\\\{\}]
{\color{incolor}In [{\color{incolor}31}]:} \PY{k}{print} \PY{n+nb}{isinstance}\PY{p}{(}\PY{l+m+mi}{1}\PY{p}{,} \PY{n+nb}{int}\PY{p}{)}
         \PY{k}{print} \PY{n+nb}{isinstance}\PY{p}{(}\PY{l+m+mf}{1.0}\PY{p}{,}\PY{n+nb}{int}\PY{p}{)}
         \PY{k}{print} \PY{n+nb}{isinstance}\PY{p}{(}\PY{l+m+mf}{1.0}\PY{p}{,}\PY{p}{(}\PY{n+nb}{int}\PY{p}{,}\PY{n+nb}{float}\PY{p}{)}\PY{p}{)}
\end{Verbatim}

    \begin{Verbatim}[commandchars=\\\{\}]
True
False
True
    \end{Verbatim}

    \textbf{cmp(x,y)}

\begin{longtable}[c]{@{}ll@{}}
\toprule
x ? y & Output\tabularnewline
\midrule
\endhead
x \textless{} y & -1\tabularnewline
x == y & 0\tabularnewline
x \textgreater{} y & 1\tabularnewline
\bottomrule
\end{longtable}

    \begin{Verbatim}[commandchars=\\\{\}]
{\color{incolor}In [{\color{incolor}32}]:} \PY{k}{print} \PY{n+nb}{cmp}\PY{p}{(}\PY{l+m+mi}{1}\PY{p}{,}\PY{l+m+mi}{2}\PY{p}{)}
         \PY{k}{print} \PY{n+nb}{cmp}\PY{p}{(}\PY{l+m+mi}{2}\PY{p}{,}\PY{l+m+mi}{1}\PY{p}{)}
         \PY{k}{print} \PY{n+nb}{cmp}\PY{p}{(}\PY{l+m+mi}{2}\PY{p}{,}\PY{l+m+mi}{2}\PY{p}{)}
\end{Verbatim}

    \begin{Verbatim}[commandchars=\\\{\}]
-1
1
0
    \end{Verbatim}

    \textbf{pow(x,y,z)} can be used to find the power \(x^y\) also the mod
of the resulting value with the third specified number can be found i.e.
: (\(x^y\) \% z).

    \begin{Verbatim}[commandchars=\\\{\}]
{\color{incolor}In [{\color{incolor}33}]:} \PY{k}{print} \PY{n+nb}{pow}\PY{p}{(}\PY{l+m+mi}{3}\PY{p}{,}\PY{l+m+mi}{3}\PY{p}{)}
         \PY{k}{print} \PY{n+nb}{pow}\PY{p}{(}\PY{l+m+mi}{3}\PY{p}{,}\PY{l+m+mi}{3}\PY{p}{,}\PY{l+m+mi}{5}\PY{p}{)}
\end{Verbatim}

    \begin{Verbatim}[commandchars=\\\{\}]
27
2
    \end{Verbatim}

    \textbf{range( )} function outputs the integers of the specified range.
It can also be used to generate a series by specifying the difference
between the two numbers within a particular range. The elements are
returned in a list (will be discussing in detail later.)

    \begin{Verbatim}[commandchars=\\\{\}]
{\color{incolor}In [{\color{incolor}34}]:} \PY{k}{print} \PY{n+nb}{range}\PY{p}{(}\PY{l+m+mi}{3}\PY{p}{)}
         \PY{k}{print} \PY{n+nb}{range}\PY{p}{(}\PY{l+m+mi}{2}\PY{p}{,}\PY{l+m+mi}{9}\PY{p}{)}
         \PY{k}{print} \PY{n+nb}{range}\PY{p}{(}\PY{l+m+mi}{2}\PY{p}{,}\PY{l+m+mi}{27}\PY{p}{,}\PY{l+m+mi}{8}\PY{p}{)}
\end{Verbatim}

    \begin{Verbatim}[commandchars=\\\{\}]
[0, 1, 2]
[2, 3, 4, 5, 6, 7, 8]
[2, 10, 18, 26]
    \end{Verbatim}

    \subsection{Accepting User Inputs}\label{accepting-user-inputs}

    \textbf{raw\_input( )} accepts input and stores it as a string. Hence,
if the user inputs a integer, the code should convert the string to an
integer and then proceed.

    \begin{Verbatim}[commandchars=\\\{\}]
{\color{incolor}In [{\color{incolor}35}]:} \PY{n}{abc} \PY{o}{=} \PY{n+nb}{raw\PYZus{}input}\PY{p}{(}\PY{l+s}{\PYZdq{}}\PY{l+s}{Type something here and it will be stored in variable abc }\PY{l+s+se}{\PYZbs{}t}\PY{l+s}{\PYZdq{}}\PY{p}{)}
\end{Verbatim}

    \begin{Verbatim}[commandchars=\\\{\}]
Type something here and it will be stored in variable abc 	Hey
    \end{Verbatim}

    \begin{Verbatim}[commandchars=\\\{\}]
{\color{incolor}In [{\color{incolor}36}]:} \PY{n+nb}{type}\PY{p}{(}\PY{n}{abc}\PY{p}{)}
\end{Verbatim}

            \begin{Verbatim}[commandchars=\\\{\}]
{\color{outcolor}Out[{\color{outcolor}36}]:} str
\end{Verbatim}
        
    \textbf{input( )}, this is used only for accepting only integer inputs.

    \begin{Verbatim}[commandchars=\\\{\}]
{\color{incolor}In [{\color{incolor}37}]:} \PY{n}{abc1} \PY{o}{=}  \PY{n+nb}{input}\PY{p}{(}\PY{l+s}{\PYZdq{}}\PY{l+s}{Only integer can be stored in in variable abc }\PY{l+s+se}{\PYZbs{}t}\PY{l+s}{\PYZdq{}}\PY{p}{)}
\end{Verbatim}

    \begin{Verbatim}[commandchars=\\\{\}]
Only integer can be stored in in variable abc 	275
    \end{Verbatim}

    \begin{Verbatim}[commandchars=\\\{\}]
{\color{incolor}In [{\color{incolor}38}]:} \PY{n+nb}{type}\PY{p}{(}\PY{n}{abc1}\PY{p}{)}
\end{Verbatim}

            \begin{Verbatim}[commandchars=\\\{\}]
{\color{outcolor}Out[{\color{outcolor}38}]:} int
\end{Verbatim}
        
    Note that \textbf{type( )} returns the format or the type of a variable
or a number


    % Add a bibliography block to the postdoc
 \newpage   
    
    
% Default to the notebook output style

    


% Inherit from the specified cell style.




    

    
    
    \definecolor{orange}{cmyk}{0,0.4,0.8,0.2}
    \definecolor{darkorange}{rgb}{.71,0.21,0.01}
    \definecolor{darkgreen}{rgb}{.12,.54,.11}
    \definecolor{myteal}{rgb}{.26, .44, .56}
    \definecolor{gray}{gray}{0.45}
    \definecolor{lightgray}{gray}{.95}
    \definecolor{mediumgray}{gray}{.8}
    \definecolor{inputbackground}{rgb}{.95, .95, .85}
    \definecolor{outputbackground}{rgb}{.95, .95, .95}
    \definecolor{traceback}{rgb}{1, .95, .95}
    % ansi colors
    \definecolor{red}{rgb}{.6,0,0}
    \definecolor{green}{rgb}{0,.65,0}
    \definecolor{brown}{rgb}{0.6,0.6,0}
    \definecolor{blue}{rgb}{0,.145,.698}
    \definecolor{purple}{rgb}{.698,.145,.698}
    \definecolor{cyan}{rgb}{0,.698,.698}
    \definecolor{lightgray}{gray}{0.5}
    
    % bright ansi colors
    \definecolor{darkgray}{gray}{0.25}
    \definecolor{lightred}{rgb}{1.0,0.39,0.28}
    \definecolor{lightgreen}{rgb}{0.48,0.99,0.0}
    \definecolor{lightblue}{rgb}{0.53,0.81,0.92}
    \definecolor{lightpurple}{rgb}{0.87,0.63,0.87}
    \definecolor{lightcyan}{rgb}{0.5,1.0,0.83}
    
    % commands and environments needed by pandoc snippets
    % extracted from the output of `pandoc -s`
    \DefineVerbatimEnvironment{Highlighting}{Verbatim}{commandchars=\\\{\}}
    % Add ',fontsize=\small' for more characters per line
    \newenvironment{Shaded}{}{}
    \newcommand{\KeywordTok}[1]{\textcolor[rgb]{0.00,0.44,0.13}{\textbf{{#1}}}}
    \newcommand{\DataTypeTok}[1]{\textcolor[rgb]{0.56,0.13,0.00}{{#1}}}
    \newcommand{\DecValTok}[1]{\textcolor[rgb]{0.25,0.63,0.44}{{#1}}}
    \newcommand{\BaseNTok}[1]{\textcolor[rgb]{0.25,0.63,0.44}{{#1}}}
    \newcommand{\FloatTok}[1]{\textcolor[rgb]{0.25,0.63,0.44}{{#1}}}
    \newcommand{\CharTok}[1]{\textcolor[rgb]{0.25,0.44,0.63}{{#1}}}
    \newcommand{\StringTok}[1]{\textcolor[rgb]{0.25,0.44,0.63}{{#1}}}
    \newcommand{\CommentTok}[1]{\textcolor[rgb]{0.38,0.63,0.69}{\textit{{#1}}}}
    \newcommand{\OtherTok}[1]{\textcolor[rgb]{0.00,0.44,0.13}{{#1}}}
    \newcommand{\AlertTok}[1]{\textcolor[rgb]{1.00,0.00,0.00}{\textbf{{#1}}}}
    \newcommand{\FunctionTok}[1]{\textcolor[rgb]{0.02,0.16,0.49}{{#1}}}
    \newcommand{\RegionMarkerTok}[1]{{#1}}
    \newcommand{\ErrorTok}[1]{\textcolor[rgb]{1.00,0.00,0.00}{\textbf{{#1}}}}
    \newcommand{\NormalTok}[1]{{#1}}
    
    % Define a nice break command that doesn't care if a line doesn't already
    % exist.
    \def\br{\hspace*{\fill} \\* }
    % Math Jax compatability definitions
    \def\gt{>}
    \def\lt{<}
    % Document parameters
    \title{}
    
    
    

    % Pygments definitions
    
\makeatletter
\def\PY@reset{\let\PY@it=\relax \let\PY@bf=\relax%
    \let\PY@ul=\relax \let\PY@tc=\relax%
    \let\PY@bc=\relax \let\PY@ff=\relax}
\def\PY@tok#1{\csname PY@tok@#1\endcsname}
\def\PY@toks#1+{\ifx\relax#1\empty\else%
    \PY@tok{#1}\expandafter\PY@toks\fi}
\def\PY@do#1{\PY@bc{\PY@tc{\PY@ul{%
    \PY@it{\PY@bf{\PY@ff{#1}}}}}}}
\def\PY#1#2{\PY@reset\PY@toks#1+\relax+\PY@do{#2}}

\expandafter\def\csname PY@tok@gd\endcsname{\def\PY@tc##1{\textcolor[rgb]{0.63,0.00,0.00}{##1}}}
\expandafter\def\csname PY@tok@gu\endcsname{\let\PY@bf=\textbf\def\PY@tc##1{\textcolor[rgb]{0.50,0.00,0.50}{##1}}}
\expandafter\def\csname PY@tok@gt\endcsname{\def\PY@tc##1{\textcolor[rgb]{0.00,0.27,0.87}{##1}}}
\expandafter\def\csname PY@tok@gs\endcsname{\let\PY@bf=\textbf}
\expandafter\def\csname PY@tok@gr\endcsname{\def\PY@tc##1{\textcolor[rgb]{1.00,0.00,0.00}{##1}}}
\expandafter\def\csname PY@tok@cm\endcsname{\let\PY@it=\textit\def\PY@tc##1{\textcolor[rgb]{0.25,0.50,0.50}{##1}}}
\expandafter\def\csname PY@tok@vg\endcsname{\def\PY@tc##1{\textcolor[rgb]{0.10,0.09,0.49}{##1}}}
\expandafter\def\csname PY@tok@m\endcsname{\def\PY@tc##1{\textcolor[rgb]{0.40,0.40,0.40}{##1}}}
\expandafter\def\csname PY@tok@mh\endcsname{\def\PY@tc##1{\textcolor[rgb]{0.40,0.40,0.40}{##1}}}
\expandafter\def\csname PY@tok@go\endcsname{\def\PY@tc##1{\textcolor[rgb]{0.53,0.53,0.53}{##1}}}
\expandafter\def\csname PY@tok@ge\endcsname{\let\PY@it=\textit}
\expandafter\def\csname PY@tok@vc\endcsname{\def\PY@tc##1{\textcolor[rgb]{0.10,0.09,0.49}{##1}}}
\expandafter\def\csname PY@tok@il\endcsname{\def\PY@tc##1{\textcolor[rgb]{0.40,0.40,0.40}{##1}}}
\expandafter\def\csname PY@tok@cs\endcsname{\let\PY@it=\textit\def\PY@tc##1{\textcolor[rgb]{0.25,0.50,0.50}{##1}}}
\expandafter\def\csname PY@tok@cp\endcsname{\def\PY@tc##1{\textcolor[rgb]{0.74,0.48,0.00}{##1}}}
\expandafter\def\csname PY@tok@gi\endcsname{\def\PY@tc##1{\textcolor[rgb]{0.00,0.63,0.00}{##1}}}
\expandafter\def\csname PY@tok@gh\endcsname{\let\PY@bf=\textbf\def\PY@tc##1{\textcolor[rgb]{0.00,0.00,0.50}{##1}}}
\expandafter\def\csname PY@tok@ni\endcsname{\let\PY@bf=\textbf\def\PY@tc##1{\textcolor[rgb]{0.60,0.60,0.60}{##1}}}
\expandafter\def\csname PY@tok@nl\endcsname{\def\PY@tc##1{\textcolor[rgb]{0.63,0.63,0.00}{##1}}}
\expandafter\def\csname PY@tok@nn\endcsname{\let\PY@bf=\textbf\def\PY@tc##1{\textcolor[rgb]{0.00,0.00,1.00}{##1}}}
\expandafter\def\csname PY@tok@no\endcsname{\def\PY@tc##1{\textcolor[rgb]{0.53,0.00,0.00}{##1}}}
\expandafter\def\csname PY@tok@na\endcsname{\def\PY@tc##1{\textcolor[rgb]{0.49,0.56,0.16}{##1}}}
\expandafter\def\csname PY@tok@nb\endcsname{\def\PY@tc##1{\textcolor[rgb]{0.00,0.50,0.00}{##1}}}
\expandafter\def\csname PY@tok@nc\endcsname{\let\PY@bf=\textbf\def\PY@tc##1{\textcolor[rgb]{0.00,0.00,1.00}{##1}}}
\expandafter\def\csname PY@tok@nd\endcsname{\def\PY@tc##1{\textcolor[rgb]{0.67,0.13,1.00}{##1}}}
\expandafter\def\csname PY@tok@ne\endcsname{\let\PY@bf=\textbf\def\PY@tc##1{\textcolor[rgb]{0.82,0.25,0.23}{##1}}}
\expandafter\def\csname PY@tok@nf\endcsname{\def\PY@tc##1{\textcolor[rgb]{0.00,0.00,1.00}{##1}}}
\expandafter\def\csname PY@tok@si\endcsname{\let\PY@bf=\textbf\def\PY@tc##1{\textcolor[rgb]{0.73,0.40,0.53}{##1}}}
\expandafter\def\csname PY@tok@s2\endcsname{\def\PY@tc##1{\textcolor[rgb]{0.73,0.13,0.13}{##1}}}
\expandafter\def\csname PY@tok@vi\endcsname{\def\PY@tc##1{\textcolor[rgb]{0.10,0.09,0.49}{##1}}}
\expandafter\def\csname PY@tok@nt\endcsname{\let\PY@bf=\textbf\def\PY@tc##1{\textcolor[rgb]{0.00,0.50,0.00}{##1}}}
\expandafter\def\csname PY@tok@nv\endcsname{\def\PY@tc##1{\textcolor[rgb]{0.10,0.09,0.49}{##1}}}
\expandafter\def\csname PY@tok@s1\endcsname{\def\PY@tc##1{\textcolor[rgb]{0.73,0.13,0.13}{##1}}}
\expandafter\def\csname PY@tok@kd\endcsname{\let\PY@bf=\textbf\def\PY@tc##1{\textcolor[rgb]{0.00,0.50,0.00}{##1}}}
\expandafter\def\csname PY@tok@sh\endcsname{\def\PY@tc##1{\textcolor[rgb]{0.73,0.13,0.13}{##1}}}
\expandafter\def\csname PY@tok@sc\endcsname{\def\PY@tc##1{\textcolor[rgb]{0.73,0.13,0.13}{##1}}}
\expandafter\def\csname PY@tok@sx\endcsname{\def\PY@tc##1{\textcolor[rgb]{0.00,0.50,0.00}{##1}}}
\expandafter\def\csname PY@tok@bp\endcsname{\def\PY@tc##1{\textcolor[rgb]{0.00,0.50,0.00}{##1}}}
\expandafter\def\csname PY@tok@c1\endcsname{\let\PY@it=\textit\def\PY@tc##1{\textcolor[rgb]{0.25,0.50,0.50}{##1}}}
\expandafter\def\csname PY@tok@kc\endcsname{\let\PY@bf=\textbf\def\PY@tc##1{\textcolor[rgb]{0.00,0.50,0.00}{##1}}}
\expandafter\def\csname PY@tok@c\endcsname{\let\PY@it=\textit\def\PY@tc##1{\textcolor[rgb]{0.25,0.50,0.50}{##1}}}
\expandafter\def\csname PY@tok@mf\endcsname{\def\PY@tc##1{\textcolor[rgb]{0.40,0.40,0.40}{##1}}}
\expandafter\def\csname PY@tok@err\endcsname{\def\PY@bc##1{\setlength{\fboxsep}{0pt}\fcolorbox[rgb]{1.00,0.00,0.00}{1,1,1}{\strut ##1}}}
\expandafter\def\csname PY@tok@mb\endcsname{\def\PY@tc##1{\textcolor[rgb]{0.40,0.40,0.40}{##1}}}
\expandafter\def\csname PY@tok@ss\endcsname{\def\PY@tc##1{\textcolor[rgb]{0.10,0.09,0.49}{##1}}}
\expandafter\def\csname PY@tok@sr\endcsname{\def\PY@tc##1{\textcolor[rgb]{0.73,0.40,0.53}{##1}}}
\expandafter\def\csname PY@tok@mo\endcsname{\def\PY@tc##1{\textcolor[rgb]{0.40,0.40,0.40}{##1}}}
\expandafter\def\csname PY@tok@kn\endcsname{\let\PY@bf=\textbf\def\PY@tc##1{\textcolor[rgb]{0.00,0.50,0.00}{##1}}}
\expandafter\def\csname PY@tok@mi\endcsname{\def\PY@tc##1{\textcolor[rgb]{0.40,0.40,0.40}{##1}}}
\expandafter\def\csname PY@tok@gp\endcsname{\let\PY@bf=\textbf\def\PY@tc##1{\textcolor[rgb]{0.00,0.00,0.50}{##1}}}
\expandafter\def\csname PY@tok@o\endcsname{\def\PY@tc##1{\textcolor[rgb]{0.40,0.40,0.40}{##1}}}
\expandafter\def\csname PY@tok@kr\endcsname{\let\PY@bf=\textbf\def\PY@tc##1{\textcolor[rgb]{0.00,0.50,0.00}{##1}}}
\expandafter\def\csname PY@tok@s\endcsname{\def\PY@tc##1{\textcolor[rgb]{0.73,0.13,0.13}{##1}}}
\expandafter\def\csname PY@tok@kp\endcsname{\def\PY@tc##1{\textcolor[rgb]{0.00,0.50,0.00}{##1}}}
\expandafter\def\csname PY@tok@w\endcsname{\def\PY@tc##1{\textcolor[rgb]{0.73,0.73,0.73}{##1}}}
\expandafter\def\csname PY@tok@kt\endcsname{\def\PY@tc##1{\textcolor[rgb]{0.69,0.00,0.25}{##1}}}
\expandafter\def\csname PY@tok@ow\endcsname{\let\PY@bf=\textbf\def\PY@tc##1{\textcolor[rgb]{0.67,0.13,1.00}{##1}}}
\expandafter\def\csname PY@tok@sb\endcsname{\def\PY@tc##1{\textcolor[rgb]{0.73,0.13,0.13}{##1}}}
\expandafter\def\csname PY@tok@k\endcsname{\let\PY@bf=\textbf\def\PY@tc##1{\textcolor[rgb]{0.00,0.50,0.00}{##1}}}
\expandafter\def\csname PY@tok@se\endcsname{\let\PY@bf=\textbf\def\PY@tc##1{\textcolor[rgb]{0.73,0.40,0.13}{##1}}}
\expandafter\def\csname PY@tok@sd\endcsname{\let\PY@it=\textit\def\PY@tc##1{\textcolor[rgb]{0.73,0.13,0.13}{##1}}}

\def\PYZbs{\char`\\}
\def\PYZus{\char`\_}
\def\PYZob{\char`\{}
\def\PYZcb{\char`\}}
\def\PYZca{\char`\^}
\def\PYZam{\char`\&}
\def\PYZlt{\char`\<}
\def\PYZgt{\char`\>}
\def\PYZsh{\char`\#}
\def\PYZpc{\char`\%}
\def\PYZdl{\char`\$}
\def\PYZhy{\char`\-}
\def\PYZsq{\char`\'}
\def\PYZdq{\char`\"}
\def\PYZti{\char`\~}
% for compatibility with earlier versions
\def\PYZat{@}
\def\PYZlb{[}
\def\PYZrb{]}
\makeatother


    % Exact colors from NB
    \definecolor{incolor}{rgb}{0.0, 0.0, 0.5}
    \definecolor{outcolor}{rgb}{0.545, 0.0, 0.0}



    
    % Prevent overflowing lines due to hard-to-break entities
    \sloppy 
    % Setup hyperref package
    \hypersetup{
      breaklinks=true,  % so long urls are correctly broken across lines
      colorlinks=true,
      urlcolor=blue,
      linkcolor=darkorange,
      citecolor=darkgreen,
      }
    % Slightly bigger margins than the latex defaults
      

    \begin{document}
    
    
    \maketitle
    
    

    
    \section{Print Statement}\label{print-statement}

    The \textbf{print} statement can be used in the following different ways
:

\begin{verbatim}
- print "Hello World"
- print "Hello", <Variable Containing the String>
- print "Hello" + <Variable Containing the String>
- print "Hello %s" % <variable containing the string>
\end{verbatim}

    \begin{Verbatim}[commandchars=\\\{\}]
{\color{incolor}In [{\color{incolor}1}]:} \PY{k}{print} \PY{l+s}{\PYZdq{}}\PY{l+s}{Hello World}\PY{l+s}{\PYZdq{}}
\end{Verbatim}

    \begin{Verbatim}[commandchars=\\\{\}]
Hello World
    \end{Verbatim}

    In Python, single, double and triple quotes are used to denote a string.
Most use single quotes when declaring a single character. Double quotes
when declaring a line and triple quotes when declaring a
paragraph/multiple lines.

    \begin{Verbatim}[commandchars=\\\{\}]
{\color{incolor}In [{\color{incolor}2}]:} \PY{k}{print} \PY{l+s}{\PYZsq{}}\PY{l+s}{Hey}\PY{l+s}{\PYZsq{}}
\end{Verbatim}

    \begin{Verbatim}[commandchars=\\\{\}]
Hey
    \end{Verbatim}

    \begin{Verbatim}[commandchars=\\\{\}]
{\color{incolor}In [{\color{incolor}3}]:} \PY{k}{print} \PY{l+s}{\PYZdq{}\PYZdq{}\PYZdq{}}\PY{l+s}{My name is Rajath Kumar M.P.}
        
        \PY{l+s}{I love Python.}\PY{l+s}{\PYZdq{}\PYZdq{}\PYZdq{}}
\end{Verbatim}

    \begin{Verbatim}[commandchars=\\\{\}]
My name is Rajath Kumar M.P.

I love Python.
    \end{Verbatim}

    Strings can be assigned to variable say \emph{string1} and
\emph{string2} which can called when using the print statement.

    \begin{Verbatim}[commandchars=\\\{\}]
{\color{incolor}In [{\color{incolor}4}]:} \PY{n}{string1} \PY{o}{=} \PY{l+s}{\PYZsq{}}\PY{l+s}{World}\PY{l+s}{\PYZsq{}}
        \PY{k}{print} \PY{l+s}{\PYZsq{}}\PY{l+s}{Hello}\PY{l+s}{\PYZsq{}}\PY{p}{,} \PY{n}{string1}
        
        \PY{n}{string2} \PY{o}{=} \PY{l+s}{\PYZsq{}}\PY{l+s}{!}\PY{l+s}{\PYZsq{}}
        \PY{k}{print} \PY{l+s}{\PYZsq{}}\PY{l+s}{Hello}\PY{l+s}{\PYZsq{}}\PY{p}{,} \PY{n}{string1}\PY{p}{,} \PY{n}{string2}
\end{Verbatim}

    \begin{Verbatim}[commandchars=\\\{\}]
Hello World
Hello World !
    \end{Verbatim}

    String concatenation is the ``addition'' of two strings. Observe that
while concatenating there will be no space between the strings.

    \begin{Verbatim}[commandchars=\\\{\}]
{\color{incolor}In [{\color{incolor}5}]:} \PY{k}{print} \PY{l+s}{\PYZsq{}}\PY{l+s}{Hello}\PY{l+s}{\PYZsq{}} \PY{o}{+} \PY{n}{string1} \PY{o}{+} \PY{n}{string2}
\end{Verbatim}

    \begin{Verbatim}[commandchars=\\\{\}]
HelloWorld!
    \end{Verbatim}

    \textbf{\%s} is used to refer to a variable which contains a string.

    \begin{Verbatim}[commandchars=\\\{\}]
{\color{incolor}In [{\color{incolor}6}]:} \PY{k}{print} \PY{l+s}{\PYZdq{}}\PY{l+s}{Hello }\PY{l+s+si}{\PYZpc{}s}\PY{l+s}{\PYZdq{}} \PY{o}{\PYZpc{}} \PY{n}{string1}
\end{Verbatim}

    \begin{Verbatim}[commandchars=\\\{\}]
Hello World
    \end{Verbatim}

    Similarly, when using other data types

\begin{verbatim}
- %s -> string
- %d -> Integer
- %f -> Float
- %o -> Octal
- %x -> Hexadecimal
- %e -> exponential
\end{verbatim}

This can be used for conversions inside the print statement itself.

    \begin{Verbatim}[commandchars=\\\{\}]
{\color{incolor}In [{\color{incolor}7}]:} \PY{k}{print} \PY{l+s}{\PYZdq{}}\PY{l+s}{Actual Number = }\PY{l+s+si}{\PYZpc{}d}\PY{l+s}{\PYZdq{}} \PY{o}{\PYZpc{}}\PY{k}{18}
        \PY{k}{print} \PY{l+s}{\PYZdq{}}\PY{l+s}{Float of the number = }\PY{l+s+si}{\PYZpc{}f}\PY{l+s}{\PYZdq{}} \PY{o}{\PYZpc{}}\PY{k}{18}
        \PY{k}{print} \PY{l+s}{\PYZdq{}}\PY{l+s}{Octal equivalent of the number = }\PY{l+s+si}{\PYZpc{}o}\PY{l+s}{\PYZdq{}} \PY{o}{\PYZpc{}}\PY{k}{18}
        \PY{k}{print} \PY{l+s}{\PYZdq{}}\PY{l+s}{Hexadecimal equivalent of the number = }\PY{l+s+si}{\PYZpc{}x}\PY{l+s}{\PYZdq{}} \PY{o}{\PYZpc{}}\PY{k}{18}
        \PY{k}{print} \PY{l+s}{\PYZdq{}}\PY{l+s}{Exponential equivalent of the number = }\PY{l+s+si}{\PYZpc{}e}\PY{l+s}{\PYZdq{}} \PY{o}{\PYZpc{}}\PY{k}{18}
\end{Verbatim}

    \begin{Verbatim}[commandchars=\\\{\}]
Actual Number = 18
Float of the number = 18.000000
Octal equivalent of the number = 22
Hexadecimal equivalent of the number = 12
Exponential equivalent of the number = 1.800000e+01
    \end{Verbatim}

    When referring to multiple variables parenthesis is used.

    \begin{Verbatim}[commandchars=\\\{\}]
{\color{incolor}In [{\color{incolor}8}]:} \PY{k}{print} \PY{l+s}{\PYZdq{}}\PY{l+s}{Hello }\PY{l+s+si}{\PYZpc{}s}\PY{l+s}{ }\PY{l+s+si}{\PYZpc{}s}\PY{l+s}{\PYZdq{}} \PY{o}{\PYZpc{}}\PY{p}{(}\PY{n}{string1}\PY{p}{,}\PY{n}{string2}\PY{p}{)}
\end{Verbatim}

    \begin{Verbatim}[commandchars=\\\{\}]
Hello World !
    \end{Verbatim}

    \subsection{Other Examples}\label{other-examples}

    The following are other different ways the print statement can be put to
use.

    \begin{Verbatim}[commandchars=\\\{\}]
{\color{incolor}In [{\color{incolor}9}]:} \PY{k}{print} \PY{l+s}{\PYZdq{}}\PY{l+s}{I want }\PY{l+s+si}{\PYZpc{}\PYZpc{}}\PY{l+s}{d to be printed }\PY{l+s+si}{\PYZpc{}s}\PY{l+s}{\PYZdq{}} \PY{o}{\PYZpc{}}\PY{l+s}{\PYZsq{}}\PY{l+s}{here}\PY{l+s}{\PYZsq{}}
\end{Verbatim}

    \begin{Verbatim}[commandchars=\\\{\}]
I want \%d to be printed here
    \end{Verbatim}

    \begin{Verbatim}[commandchars=\\\{\}]
{\color{incolor}In [{\color{incolor}10}]:} \PY{k}{print} \PY{l+s}{\PYZsq{}}\PY{l+s}{\PYZus{}A}\PY{l+s}{\PYZsq{}}\PY{o}{*}\PY{l+m+mi}{10}
\end{Verbatim}

    \begin{Verbatim}[commandchars=\\\{\}]
\_A\_A\_A\_A\_A\_A\_A\_A\_A\_A
    \end{Verbatim}

    \begin{Verbatim}[commandchars=\\\{\}]
{\color{incolor}In [{\color{incolor}11}]:} \PY{k}{print} \PY{l+s}{\PYZdq{}}\PY{l+s}{Jan}\PY{l+s+se}{\PYZbs{}n}\PY{l+s}{Feb}\PY{l+s+se}{\PYZbs{}n}\PY{l+s}{Mar}\PY{l+s+se}{\PYZbs{}n}\PY{l+s}{Apr}\PY{l+s+se}{\PYZbs{}n}\PY{l+s}{May}\PY{l+s+se}{\PYZbs{}n}\PY{l+s}{Jun}\PY{l+s+se}{\PYZbs{}n}\PY{l+s}{Jul}\PY{l+s+se}{\PYZbs{}n}\PY{l+s}{Aug}\PY{l+s}{\PYZdq{}}
\end{Verbatim}

    \begin{Verbatim}[commandchars=\\\{\}]
Jan
Feb
Mar
Apr
May
Jun
Jul
Aug
    \end{Verbatim}

    \begin{Verbatim}[commandchars=\\\{\}]
{\color{incolor}In [{\color{incolor}12}]:} \PY{k}{print} \PY{l+s}{\PYZdq{}}\PY{l+s}{I want }\PY{l+s+se}{\PYZbs{}\PYZbs{}}\PY{l+s}{n to be printed.}\PY{l+s}{\PYZdq{}}
\end{Verbatim}

    \begin{Verbatim}[commandchars=\\\{\}]
I want \textbackslash{}n to be printed.
    \end{Verbatim}

    \begin{Verbatim}[commandchars=\\\{\}]
{\color{incolor}In [{\color{incolor}13}]:} \PY{k}{print} \PY{l+s}{\PYZdq{}\PYZdq{}\PYZdq{}}
         \PY{l+s}{Routine:}
         \PY{l+s+se}{\PYZbs{}t}\PY{l+s}{\PYZhy{} Eat}
         \PY{l+s+se}{\PYZbs{}t}\PY{l+s}{\PYZhy{} Sleep}\PY{l+s+se}{\PYZbs{}n}\PY{l+s+se}{\PYZbs{}t}\PY{l+s}{\PYZhy{} Repeat}
         \PY{l+s}{\PYZdq{}\PYZdq{}\PYZdq{}}
\end{Verbatim}

    \begin{Verbatim}[commandchars=\\\{\}]
Routine:
	- Eat
	- Sleep
	- Repeat
    \end{Verbatim}

    \section{PrecisionWidth and
FieldWidth}\label{precisionwidth-and-fieldwidth}

    Fieldwidth is the width of the entire number and precision is the width
towards the right. One can alter these widths based on the requirements.

The default Precision Width is set to 6.

    \begin{Verbatim}[commandchars=\\\{\}]
{\color{incolor}In [{\color{incolor}14}]:} \PY{l+s}{\PYZdq{}}\PY{l+s+si}{\PYZpc{}f}\PY{l+s}{\PYZdq{}} \PY{o}{\PYZpc{}} \PY{l+m+mf}{3.121312312312}
\end{Verbatim}

            \begin{Verbatim}[commandchars=\\\{\}]
{\color{outcolor}Out[{\color{outcolor}14}]:} '3.121312'
\end{Verbatim}
        
    Notice upto 6 decimal points are returned. To specify the number of
decimal points, `\%(fieldwidth).(precisionwidth)f' is used.

    \begin{Verbatim}[commandchars=\\\{\}]
{\color{incolor}In [{\color{incolor}15}]:} \PY{l+s}{\PYZdq{}}\PY{l+s+si}{\PYZpc{}.5f}\PY{l+s}{\PYZdq{}} \PY{o}{\PYZpc{}} \PY{l+m+mf}{3.121312312312}
\end{Verbatim}

            \begin{Verbatim}[commandchars=\\\{\}]
{\color{outcolor}Out[{\color{outcolor}15}]:} '3.12131'
\end{Verbatim}
        
    If the field width is set more than the necessary than the data right
aligns itself to adjust to the specified values.

    \begin{Verbatim}[commandchars=\\\{\}]
{\color{incolor}In [{\color{incolor}16}]:} \PY{l+s}{\PYZdq{}}\PY{l+s+si}{\PYZpc{}9.5f}\PY{l+s}{\PYZdq{}} \PY{o}{\PYZpc{}} \PY{l+m+mf}{3.121312312312}
\end{Verbatim}

            \begin{Verbatim}[commandchars=\\\{\}]
{\color{outcolor}Out[{\color{outcolor}16}]:} '  3.12131'
\end{Verbatim}
        
    Zero padding is done by adding a 0 at the start of fieldwidth.

    \begin{Verbatim}[commandchars=\\\{\}]
{\color{incolor}In [{\color{incolor}17}]:} \PY{l+s}{\PYZdq{}}\PY{l+s+si}{\PYZpc{}020.5f}\PY{l+s}{\PYZdq{}} \PY{o}{\PYZpc{}} \PY{l+m+mf}{3.121312312312}
\end{Verbatim}

            \begin{Verbatim}[commandchars=\\\{\}]
{\color{outcolor}Out[{\color{outcolor}17}]:} '00000000000003.12131'
\end{Verbatim}
        
    For proper alignment, a space can be left blank in the field width so
that when a negative number is used, proper alignment is maintained.

    \begin{Verbatim}[commandchars=\\\{\}]
{\color{incolor}In [{\color{incolor}18}]:} \PY{k}{print} \PY{l+s}{\PYZdq{}}\PY{l+s+si}{\PYZpc{} 9f}\PY{l+s}{\PYZdq{}} \PY{o}{\PYZpc{}} \PY{l+m+mf}{3.121312312312}
         \PY{k}{print} \PY{l+s}{\PYZdq{}}\PY{l+s+si}{\PYZpc{} 9f}\PY{l+s}{\PYZdq{}} \PY{o}{\PYZpc{}} \PY{o}{\PYZhy{}}\PY{l+m+mf}{3.121312312312}
\end{Verbatim}

    \begin{Verbatim}[commandchars=\\\{\}]
3.121312
-3.121312
    \end{Verbatim}

    `+' sign can be returned at the beginning of a positive number by adding
a + sign at the beginning of the field width.

    \begin{Verbatim}[commandchars=\\\{\}]
{\color{incolor}In [{\color{incolor}19}]:} \PY{k}{print} \PY{l+s}{\PYZdq{}}\PY{l+s+si}{\PYZpc{}+9f}\PY{l+s}{\PYZdq{}} \PY{o}{\PYZpc{}} \PY{l+m+mf}{3.121312312312}
         \PY{k}{print} \PY{l+s}{\PYZdq{}}\PY{l+s+si}{\PYZpc{} 9f}\PY{l+s}{\PYZdq{}} \PY{o}{\PYZpc{}} \PY{o}{\PYZhy{}}\PY{l+m+mf}{3.121312312312}
\end{Verbatim}

    \begin{Verbatim}[commandchars=\\\{\}]
+3.121312
-3.121312
    \end{Verbatim}

    As mentioned above, the data right aligns itself when the field width
mentioned is larger than the actualy field width. But left alignment can
be done by specifying a negative symbol in the field width.

    \begin{Verbatim}[commandchars=\\\{\}]
{\color{incolor}In [{\color{incolor}20}]:} \PY{l+s}{\PYZdq{}}\PY{l+s+si}{\PYZpc{}\PYZhy{}9.3f}\PY{l+s}{\PYZdq{}} \PY{o}{\PYZpc{}} \PY{l+m+mf}{3.121312312312}
\end{Verbatim}

            \begin{Verbatim}[commandchars=\\\{\}]
{\color{outcolor}Out[{\color{outcolor}20}]:} '3.121    '
\end{Verbatim}
        

    % Add a bibliography block to the postdoc
  \newpage
  
  
% Default to the notebook output style

    


% Inherit from the specified cell style.




    

    
    
    \definecolor{orange}{cmyk}{0,0.4,0.8,0.2}
    \definecolor{darkorange}{rgb}{.71,0.21,0.01}
    \definecolor{darkgreen}{rgb}{.12,.54,.11}
    \definecolor{myteal}{rgb}{.26, .44, .56}
    \definecolor{gray}{gray}{0.45}
    \definecolor{lightgray}{gray}{.95}
    \definecolor{mediumgray}{gray}{.8}
    \definecolor{inputbackground}{rgb}{.95, .95, .85}
    \definecolor{outputbackground}{rgb}{.95, .95, .95}
    \definecolor{traceback}{rgb}{1, .95, .95}
    % ansi colors
    \definecolor{red}{rgb}{.6,0,0}
    \definecolor{green}{rgb}{0,.65,0}
    \definecolor{brown}{rgb}{0.6,0.6,0}
    \definecolor{blue}{rgb}{0,.145,.698}
    \definecolor{purple}{rgb}{.698,.145,.698}
    \definecolor{cyan}{rgb}{0,.698,.698}
    \definecolor{lightgray}{gray}{0.5}
    
    % bright ansi colors
    \definecolor{darkgray}{gray}{0.25}
    \definecolor{lightred}{rgb}{1.0,0.39,0.28}
    \definecolor{lightgreen}{rgb}{0.48,0.99,0.0}
    \definecolor{lightblue}{rgb}{0.53,0.81,0.92}
    \definecolor{lightpurple}{rgb}{0.87,0.63,0.87}
    \definecolor{lightcyan}{rgb}{0.5,1.0,0.83}
    
    % commands and environments needed by pandoc snippets
    % extracted from the output of `pandoc -s`
    \DefineVerbatimEnvironment{Highlighting}{Verbatim}{commandchars=\\\{\}}
    % Add ',fontsize=\small' for more characters per line
    \newenvironment{Shaded}{}{}
    \newcommand{\KeywordTok}[1]{\textcolor[rgb]{0.00,0.44,0.13}{\textbf{{#1}}}}
    \newcommand{\DataTypeTok}[1]{\textcolor[rgb]{0.56,0.13,0.00}{{#1}}}
    \newcommand{\DecValTok}[1]{\textcolor[rgb]{0.25,0.63,0.44}{{#1}}}
    \newcommand{\BaseNTok}[1]{\textcolor[rgb]{0.25,0.63,0.44}{{#1}}}
    \newcommand{\FloatTok}[1]{\textcolor[rgb]{0.25,0.63,0.44}{{#1}}}
    \newcommand{\CharTok}[1]{\textcolor[rgb]{0.25,0.44,0.63}{{#1}}}
    \newcommand{\StringTok}[1]{\textcolor[rgb]{0.25,0.44,0.63}{{#1}}}
    \newcommand{\CommentTok}[1]{\textcolor[rgb]{0.38,0.63,0.69}{\textit{{#1}}}}
    \newcommand{\OtherTok}[1]{\textcolor[rgb]{0.00,0.44,0.13}{{#1}}}
    \newcommand{\AlertTok}[1]{\textcolor[rgb]{1.00,0.00,0.00}{\textbf{{#1}}}}
    \newcommand{\FunctionTok}[1]{\textcolor[rgb]{0.02,0.16,0.49}{{#1}}}
    \newcommand{\RegionMarkerTok}[1]{{#1}}
    \newcommand{\ErrorTok}[1]{\textcolor[rgb]{1.00,0.00,0.00}{\textbf{{#1}}}}
    \newcommand{\NormalTok}[1]{{#1}}
    
    % Define a nice break command that doesn't care if a line doesn't already
    % exist.
    \def\br{\hspace*{\fill} \\* }
    % Math Jax compatability definitions
    \def\gt{>}
    \def\lt{<}
    % Document parameters
    \title{}
    
    
    

    % Pygments definitions
    
\makeatletter
\def\PY@reset{\let\PY@it=\relax \let\PY@bf=\relax%
    \let\PY@ul=\relax \let\PY@tc=\relax%
    \let\PY@bc=\relax \let\PY@ff=\relax}
\def\PY@tok#1{\csname PY@tok@#1\endcsname}
\def\PY@toks#1+{\ifx\relax#1\empty\else%
    \PY@tok{#1}\expandafter\PY@toks\fi}
\def\PY@do#1{\PY@bc{\PY@tc{\PY@ul{%
    \PY@it{\PY@bf{\PY@ff{#1}}}}}}}
\def\PY#1#2{\PY@reset\PY@toks#1+\relax+\PY@do{#2}}

\expandafter\def\csname PY@tok@gd\endcsname{\def\PY@tc##1{\textcolor[rgb]{0.63,0.00,0.00}{##1}}}
\expandafter\def\csname PY@tok@gu\endcsname{\let\PY@bf=\textbf\def\PY@tc##1{\textcolor[rgb]{0.50,0.00,0.50}{##1}}}
\expandafter\def\csname PY@tok@gt\endcsname{\def\PY@tc##1{\textcolor[rgb]{0.00,0.27,0.87}{##1}}}
\expandafter\def\csname PY@tok@gs\endcsname{\let\PY@bf=\textbf}
\expandafter\def\csname PY@tok@gr\endcsname{\def\PY@tc##1{\textcolor[rgb]{1.00,0.00,0.00}{##1}}}
\expandafter\def\csname PY@tok@cm\endcsname{\let\PY@it=\textit\def\PY@tc##1{\textcolor[rgb]{0.25,0.50,0.50}{##1}}}
\expandafter\def\csname PY@tok@vg\endcsname{\def\PY@tc##1{\textcolor[rgb]{0.10,0.09,0.49}{##1}}}
\expandafter\def\csname PY@tok@m\endcsname{\def\PY@tc##1{\textcolor[rgb]{0.40,0.40,0.40}{##1}}}
\expandafter\def\csname PY@tok@mh\endcsname{\def\PY@tc##1{\textcolor[rgb]{0.40,0.40,0.40}{##1}}}
\expandafter\def\csname PY@tok@go\endcsname{\def\PY@tc##1{\textcolor[rgb]{0.53,0.53,0.53}{##1}}}
\expandafter\def\csname PY@tok@ge\endcsname{\let\PY@it=\textit}
\expandafter\def\csname PY@tok@vc\endcsname{\def\PY@tc##1{\textcolor[rgb]{0.10,0.09,0.49}{##1}}}
\expandafter\def\csname PY@tok@il\endcsname{\def\PY@tc##1{\textcolor[rgb]{0.40,0.40,0.40}{##1}}}
\expandafter\def\csname PY@tok@cs\endcsname{\let\PY@it=\textit\def\PY@tc##1{\textcolor[rgb]{0.25,0.50,0.50}{##1}}}
\expandafter\def\csname PY@tok@cp\endcsname{\def\PY@tc##1{\textcolor[rgb]{0.74,0.48,0.00}{##1}}}
\expandafter\def\csname PY@tok@gi\endcsname{\def\PY@tc##1{\textcolor[rgb]{0.00,0.63,0.00}{##1}}}
\expandafter\def\csname PY@tok@gh\endcsname{\let\PY@bf=\textbf\def\PY@tc##1{\textcolor[rgb]{0.00,0.00,0.50}{##1}}}
\expandafter\def\csname PY@tok@ni\endcsname{\let\PY@bf=\textbf\def\PY@tc##1{\textcolor[rgb]{0.60,0.60,0.60}{##1}}}
\expandafter\def\csname PY@tok@nl\endcsname{\def\PY@tc##1{\textcolor[rgb]{0.63,0.63,0.00}{##1}}}
\expandafter\def\csname PY@tok@nn\endcsname{\let\PY@bf=\textbf\def\PY@tc##1{\textcolor[rgb]{0.00,0.00,1.00}{##1}}}
\expandafter\def\csname PY@tok@no\endcsname{\def\PY@tc##1{\textcolor[rgb]{0.53,0.00,0.00}{##1}}}
\expandafter\def\csname PY@tok@na\endcsname{\def\PY@tc##1{\textcolor[rgb]{0.49,0.56,0.16}{##1}}}
\expandafter\def\csname PY@tok@nb\endcsname{\def\PY@tc##1{\textcolor[rgb]{0.00,0.50,0.00}{##1}}}
\expandafter\def\csname PY@tok@nc\endcsname{\let\PY@bf=\textbf\def\PY@tc##1{\textcolor[rgb]{0.00,0.00,1.00}{##1}}}
\expandafter\def\csname PY@tok@nd\endcsname{\def\PY@tc##1{\textcolor[rgb]{0.67,0.13,1.00}{##1}}}
\expandafter\def\csname PY@tok@ne\endcsname{\let\PY@bf=\textbf\def\PY@tc##1{\textcolor[rgb]{0.82,0.25,0.23}{##1}}}
\expandafter\def\csname PY@tok@nf\endcsname{\def\PY@tc##1{\textcolor[rgb]{0.00,0.00,1.00}{##1}}}
\expandafter\def\csname PY@tok@si\endcsname{\let\PY@bf=\textbf\def\PY@tc##1{\textcolor[rgb]{0.73,0.40,0.53}{##1}}}
\expandafter\def\csname PY@tok@s2\endcsname{\def\PY@tc##1{\textcolor[rgb]{0.73,0.13,0.13}{##1}}}
\expandafter\def\csname PY@tok@vi\endcsname{\def\PY@tc##1{\textcolor[rgb]{0.10,0.09,0.49}{##1}}}
\expandafter\def\csname PY@tok@nt\endcsname{\let\PY@bf=\textbf\def\PY@tc##1{\textcolor[rgb]{0.00,0.50,0.00}{##1}}}
\expandafter\def\csname PY@tok@nv\endcsname{\def\PY@tc##1{\textcolor[rgb]{0.10,0.09,0.49}{##1}}}
\expandafter\def\csname PY@tok@s1\endcsname{\def\PY@tc##1{\textcolor[rgb]{0.73,0.13,0.13}{##1}}}
\expandafter\def\csname PY@tok@kd\endcsname{\let\PY@bf=\textbf\def\PY@tc##1{\textcolor[rgb]{0.00,0.50,0.00}{##1}}}
\expandafter\def\csname PY@tok@sh\endcsname{\def\PY@tc##1{\textcolor[rgb]{0.73,0.13,0.13}{##1}}}
\expandafter\def\csname PY@tok@sc\endcsname{\def\PY@tc##1{\textcolor[rgb]{0.73,0.13,0.13}{##1}}}
\expandafter\def\csname PY@tok@sx\endcsname{\def\PY@tc##1{\textcolor[rgb]{0.00,0.50,0.00}{##1}}}
\expandafter\def\csname PY@tok@bp\endcsname{\def\PY@tc##1{\textcolor[rgb]{0.00,0.50,0.00}{##1}}}
\expandafter\def\csname PY@tok@c1\endcsname{\let\PY@it=\textit\def\PY@tc##1{\textcolor[rgb]{0.25,0.50,0.50}{##1}}}
\expandafter\def\csname PY@tok@kc\endcsname{\let\PY@bf=\textbf\def\PY@tc##1{\textcolor[rgb]{0.00,0.50,0.00}{##1}}}
\expandafter\def\csname PY@tok@c\endcsname{\let\PY@it=\textit\def\PY@tc##1{\textcolor[rgb]{0.25,0.50,0.50}{##1}}}
\expandafter\def\csname PY@tok@mf\endcsname{\def\PY@tc##1{\textcolor[rgb]{0.40,0.40,0.40}{##1}}}
\expandafter\def\csname PY@tok@err\endcsname{\def\PY@bc##1{\setlength{\fboxsep}{0pt}\fcolorbox[rgb]{1.00,0.00,0.00}{1,1,1}{\strut ##1}}}
\expandafter\def\csname PY@tok@mb\endcsname{\def\PY@tc##1{\textcolor[rgb]{0.40,0.40,0.40}{##1}}}
\expandafter\def\csname PY@tok@ss\endcsname{\def\PY@tc##1{\textcolor[rgb]{0.10,0.09,0.49}{##1}}}
\expandafter\def\csname PY@tok@sr\endcsname{\def\PY@tc##1{\textcolor[rgb]{0.73,0.40,0.53}{##1}}}
\expandafter\def\csname PY@tok@mo\endcsname{\def\PY@tc##1{\textcolor[rgb]{0.40,0.40,0.40}{##1}}}
\expandafter\def\csname PY@tok@kn\endcsname{\let\PY@bf=\textbf\def\PY@tc##1{\textcolor[rgb]{0.00,0.50,0.00}{##1}}}
\expandafter\def\csname PY@tok@mi\endcsname{\def\PY@tc##1{\textcolor[rgb]{0.40,0.40,0.40}{##1}}}
\expandafter\def\csname PY@tok@gp\endcsname{\let\PY@bf=\textbf\def\PY@tc##1{\textcolor[rgb]{0.00,0.00,0.50}{##1}}}
\expandafter\def\csname PY@tok@o\endcsname{\def\PY@tc##1{\textcolor[rgb]{0.40,0.40,0.40}{##1}}}
\expandafter\def\csname PY@tok@kr\endcsname{\let\PY@bf=\textbf\def\PY@tc##1{\textcolor[rgb]{0.00,0.50,0.00}{##1}}}
\expandafter\def\csname PY@tok@s\endcsname{\def\PY@tc##1{\textcolor[rgb]{0.73,0.13,0.13}{##1}}}
\expandafter\def\csname PY@tok@kp\endcsname{\def\PY@tc##1{\textcolor[rgb]{0.00,0.50,0.00}{##1}}}
\expandafter\def\csname PY@tok@w\endcsname{\def\PY@tc##1{\textcolor[rgb]{0.73,0.73,0.73}{##1}}}
\expandafter\def\csname PY@tok@kt\endcsname{\def\PY@tc##1{\textcolor[rgb]{0.69,0.00,0.25}{##1}}}
\expandafter\def\csname PY@tok@ow\endcsname{\let\PY@bf=\textbf\def\PY@tc##1{\textcolor[rgb]{0.67,0.13,1.00}{##1}}}
\expandafter\def\csname PY@tok@sb\endcsname{\def\PY@tc##1{\textcolor[rgb]{0.73,0.13,0.13}{##1}}}
\expandafter\def\csname PY@tok@k\endcsname{\let\PY@bf=\textbf\def\PY@tc##1{\textcolor[rgb]{0.00,0.50,0.00}{##1}}}
\expandafter\def\csname PY@tok@se\endcsname{\let\PY@bf=\textbf\def\PY@tc##1{\textcolor[rgb]{0.73,0.40,0.13}{##1}}}
\expandafter\def\csname PY@tok@sd\endcsname{\let\PY@it=\textit\def\PY@tc##1{\textcolor[rgb]{0.73,0.13,0.13}{##1}}}

\def\PYZbs{\char`\\}
\def\PYZus{\char`\_}
\def\PYZob{\char`\{}
\def\PYZcb{\char`\}}
\def\PYZca{\char`\^}
\def\PYZam{\char`\&}
\def\PYZlt{\char`\<}
\def\PYZgt{\char`\>}
\def\PYZsh{\char`\#}
\def\PYZpc{\char`\%}
\def\PYZdl{\char`\$}
\def\PYZhy{\char`\-}
\def\PYZsq{\char`\'}
\def\PYZdq{\char`\"}
\def\PYZti{\char`\~}
% for compatibility with earlier versions
\def\PYZat{@}
\def\PYZlb{[}
\def\PYZrb{]}
\makeatother


    % Exact colors from NB
    \definecolor{incolor}{rgb}{0.0, 0.0, 0.5}
    \definecolor{outcolor}{rgb}{0.545, 0.0, 0.0}



    
    % Prevent overflowing lines due to hard-to-break entities
    \sloppy 
    % Setup hyperref package
    \hypersetup{
      breaklinks=true,  % so long urls are correctly broken across lines
      colorlinks=true,
      urlcolor=blue,
      linkcolor=darkorange,
      citecolor=darkgreen,
      }
    % Slightly bigger margins than the latex defaults
    
   
    
    

    \begin{document}
    
    
    \maketitle
    
    

    
    \section{Data Structures}\label{data-structures}

    In simple terms, It is the the collection or group of data in a
particular structure.

    \subsection{Lists}\label{lists}

    Lists are the most commonly used data structure. Think of it as a
sequence of data that is enclosed in square brackets and data are
separated by a comma. Each of these data can be accessed by calling it's
index value.

Lists are declared by just equating a variable to `{[} {]}' or list.

    \begin{Verbatim}[commandchars=\\\{\}]
{\color{incolor}In [{\color{incolor}1}]:} \PY{n}{a} \PY{o}{=} \PY{p}{[}\PY{p}{]}
\end{Verbatim}

    \begin{Verbatim}[commandchars=\\\{\}]
{\color{incolor}In [{\color{incolor}2}]:} \PY{k}{print} \PY{n+nb}{type}\PY{p}{(}\PY{n}{a}\PY{p}{)}
\end{Verbatim}

    \begin{Verbatim}[commandchars=\\\{\}]
<type 'list'>
    \end{Verbatim}

    One can directly assign the sequence of data to a list x as shown.

    \begin{Verbatim}[commandchars=\\\{\}]
{\color{incolor}In [{\color{incolor}3}]:} \PY{n}{x} \PY{o}{=} \PY{p}{[}\PY{l+s}{\PYZsq{}}\PY{l+s}{apple}\PY{l+s}{\PYZsq{}}\PY{p}{,} \PY{l+s}{\PYZsq{}}\PY{l+s}{orange}\PY{l+s}{\PYZsq{}}\PY{p}{]}
\end{Verbatim}

    \subsubsection{Indexing}\label{indexing}

    In python, Indexing starts from 0. Thus now the list x, which has two
elements will have apple at 0 index and orange at 1 index.

    \begin{Verbatim}[commandchars=\\\{\}]
{\color{incolor}In [{\color{incolor}4}]:} \PY{n}{x}\PY{p}{[}\PY{l+m+mi}{0}\PY{p}{]}
\end{Verbatim}

            \begin{Verbatim}[commandchars=\\\{\}]
{\color{outcolor}Out[{\color{outcolor}4}]:} 'apple'
\end{Verbatim}
        
    Indexing can also be done in reverse order. That is the last element can
be accessed first. Here, indexing starts from -1. Thus index value -1
will be orange and index -2 will be apple.

    \begin{Verbatim}[commandchars=\\\{\}]
{\color{incolor}In [{\color{incolor}5}]:} \PY{n}{x}\PY{p}{[}\PY{o}{\PYZhy{}}\PY{l+m+mi}{1}\PY{p}{]}
\end{Verbatim}

            \begin{Verbatim}[commandchars=\\\{\}]
{\color{outcolor}Out[{\color{outcolor}5}]:} 'orange'
\end{Verbatim}
        
    As you might have already guessed, x{[}0{]} = x{[}-2{]}, x{[}1{]} =
x{[}-1{]}. This concept can be extended towards lists with more many
elements.

    \begin{Verbatim}[commandchars=\\\{\}]
{\color{incolor}In [{\color{incolor}6}]:} \PY{n}{y} \PY{o}{=} \PY{p}{[}\PY{l+s}{\PYZsq{}}\PY{l+s}{carrot}\PY{l+s}{\PYZsq{}}\PY{p}{,}\PY{l+s}{\PYZsq{}}\PY{l+s}{potato}\PY{l+s}{\PYZsq{}}\PY{p}{]}
\end{Verbatim}

    Here we have declared two lists x and y each containing its own data.
Now, these two lists can again be put into another list say z which will
have it's data as two lists. This list inside a list is called as nested
lists and is how an array would be declared which we will see later.

    \begin{Verbatim}[commandchars=\\\{\}]
{\color{incolor}In [{\color{incolor}7}]:} \PY{n}{z}  \PY{o}{=} \PY{p}{[}\PY{n}{x}\PY{p}{,}\PY{n}{y}\PY{p}{]}
        \PY{k}{print} \PY{n}{z}
\end{Verbatim}

    \begin{Verbatim}[commandchars=\\\{\}]
[['apple', 'orange'], ['carrot', 'potato']]
    \end{Verbatim}

    Indexing in nested lists can be quite confusing if you do not understand
how indexing works in python. So let us break it down and then arrive at
a conclusion.

Let us access the data `apple' in the above nested list. First, at index
0 there is a list {[}`apple',`orange'{]} and at index 1 there is another
list {[}`carrot',`potato'{]}. Hence z{[}0{]} should give us the first
list which contains `apple'.

    \begin{Verbatim}[commandchars=\\\{\}]
{\color{incolor}In [{\color{incolor}8}]:} \PY{n}{z1} \PY{o}{=} \PY{n}{z}\PY{p}{[}\PY{l+m+mi}{0}\PY{p}{]}
        \PY{k}{print} \PY{n}{z1}
\end{Verbatim}

    \begin{Verbatim}[commandchars=\\\{\}]
['apple', 'orange']
    \end{Verbatim}

    Now observe that z1 is not at all a nested list thus to access `apple',
z1 should be indexed at 0.

    \begin{Verbatim}[commandchars=\\\{\}]
{\color{incolor}In [{\color{incolor}9}]:} \PY{n}{z1}\PY{p}{[}\PY{l+m+mi}{0}\PY{p}{]}
\end{Verbatim}

            \begin{Verbatim}[commandchars=\\\{\}]
{\color{outcolor}Out[{\color{outcolor}9}]:} 'apple'
\end{Verbatim}
        
    Instead of doing the above, In python, you can access `apple' by just
writing the index values each time side by side.

    \begin{Verbatim}[commandchars=\\\{\}]
{\color{incolor}In [{\color{incolor}10}]:} \PY{n}{z}\PY{p}{[}\PY{l+m+mi}{0}\PY{p}{]}\PY{p}{[}\PY{l+m+mi}{0}\PY{p}{]}
\end{Verbatim}

            \begin{Verbatim}[commandchars=\\\{\}]
{\color{outcolor}Out[{\color{outcolor}10}]:} 'apple'
\end{Verbatim}
        
    If there was a list inside a list inside a list then you can access the
innermost value by executing z{[} {]}{[} {]}{[} {]}.

    \subsubsection{Slicing}\label{slicing}

    Indexing was only limited to accessing a single element, Slicing on the
other hand is accessing a sequence of data inside the list. In other
words ``slicing'' the list.

Slicing is done by defining the index values of the first element and
the last element from the parent list that is required in the sliced
list. It is written as parentlist{[} a : b {]} where a,b are the index
values from the parent list. If a or b is not defined then the index
value is considered to be the first value for a if a is not defined and
the last value for b when b is not defined.

    \begin{Verbatim}[commandchars=\\\{\}]
{\color{incolor}In [{\color{incolor}11}]:} \PY{n}{num} \PY{o}{=} \PY{p}{[}\PY{l+m+mi}{0}\PY{p}{,}\PY{l+m+mi}{1}\PY{p}{,}\PY{l+m+mi}{2}\PY{p}{,}\PY{l+m+mi}{3}\PY{p}{,}\PY{l+m+mi}{4}\PY{p}{,}\PY{l+m+mi}{5}\PY{p}{,}\PY{l+m+mi}{6}\PY{p}{,}\PY{l+m+mi}{7}\PY{p}{,}\PY{l+m+mi}{8}\PY{p}{,}\PY{l+m+mi}{9}\PY{p}{]}
\end{Verbatim}

    \begin{Verbatim}[commandchars=\\\{\}]
{\color{incolor}In [{\color{incolor}12}]:} \PY{k}{print} \PY{n}{num}\PY{p}{[}\PY{l+m+mi}{0}\PY{p}{:}\PY{l+m+mi}{4}\PY{p}{]}
         \PY{k}{print} \PY{n}{num}\PY{p}{[}\PY{l+m+mi}{4}\PY{p}{:}\PY{p}{]}
\end{Verbatim}

    \begin{Verbatim}[commandchars=\\\{\}]
[0, 1, 2, 3]
[4, 5, 6, 7, 8, 9]
    \end{Verbatim}

    You can also slice a parent list with a fixed length or step length.

    \begin{Verbatim}[commandchars=\\\{\}]
{\color{incolor}In [{\color{incolor}13}]:} \PY{n}{num}\PY{p}{[}\PY{p}{:}\PY{l+m+mi}{9}\PY{p}{:}\PY{l+m+mi}{3}\PY{p}{]}
\end{Verbatim}

            \begin{Verbatim}[commandchars=\\\{\}]
{\color{outcolor}Out[{\color{outcolor}13}]:} [0, 3, 6]
\end{Verbatim}
        
    \subsubsection{Built in List Functions}\label{built-in-list-functions}

    To find the length of the list or the number of elements in a list,
\textbf{len( )} is used.

    \begin{Verbatim}[commandchars=\\\{\}]
{\color{incolor}In [{\color{incolor}14}]:} \PY{n+nb}{len}\PY{p}{(}\PY{n}{num}\PY{p}{)}
\end{Verbatim}

            \begin{Verbatim}[commandchars=\\\{\}]
{\color{outcolor}Out[{\color{outcolor}14}]:} 10
\end{Verbatim}
        
    If the list consists of all integer elements then \textbf{min( )} and
\textbf{max( )} gives the minimum and maximum value in the list.

    \begin{Verbatim}[commandchars=\\\{\}]
{\color{incolor}In [{\color{incolor}15}]:} \PY{n+nb}{min}\PY{p}{(}\PY{n}{num}\PY{p}{)}
\end{Verbatim}

            \begin{Verbatim}[commandchars=\\\{\}]
{\color{outcolor}Out[{\color{outcolor}15}]:} 0
\end{Verbatim}
        
    \begin{Verbatim}[commandchars=\\\{\}]
{\color{incolor}In [{\color{incolor}16}]:} \PY{n+nb}{max}\PY{p}{(}\PY{n}{num}\PY{p}{)}
\end{Verbatim}

            \begin{Verbatim}[commandchars=\\\{\}]
{\color{outcolor}Out[{\color{outcolor}16}]:} 9
\end{Verbatim}
        
    Lists can be concatenated by adding, `+' them. The resultant list will
contain all the elements of the lists that were added. The resultant
list will not be a nested list.

    \begin{Verbatim}[commandchars=\\\{\}]
{\color{incolor}In [{\color{incolor}17}]:} \PY{p}{[}\PY{l+m+mi}{1}\PY{p}{,}\PY{l+m+mi}{2}\PY{p}{,}\PY{l+m+mi}{3}\PY{p}{]} \PY{o}{+} \PY{p}{[}\PY{l+m+mi}{5}\PY{p}{,}\PY{l+m+mi}{4}\PY{p}{,}\PY{l+m+mi}{7}\PY{p}{]}
\end{Verbatim}

            \begin{Verbatim}[commandchars=\\\{\}]
{\color{outcolor}Out[{\color{outcolor}17}]:} [1, 2, 3, 5, 4, 7]
\end{Verbatim}
        
    There might arise a requirement where you might need to check if a
particular element is there in a predefined list. Consider the below
list.

    \begin{Verbatim}[commandchars=\\\{\}]
{\color{incolor}In [{\color{incolor}18}]:} \PY{n}{names} \PY{o}{=} \PY{p}{[}\PY{l+s}{\PYZsq{}}\PY{l+s}{Earth}\PY{l+s}{\PYZsq{}}\PY{p}{,}\PY{l+s}{\PYZsq{}}\PY{l+s}{Air}\PY{l+s}{\PYZsq{}}\PY{p}{,}\PY{l+s}{\PYZsq{}}\PY{l+s}{Fire}\PY{l+s}{\PYZsq{}}\PY{p}{,}\PY{l+s}{\PYZsq{}}\PY{l+s}{Water}\PY{l+s}{\PYZsq{}}\PY{p}{]}
\end{Verbatim}

    To check if `Fire' and `Rajath' is present in the list names. A
conventional approach would be to use a for loop and iterate over the
list and use the if condition. But in python you can use `a in b'
concept which would return `True' if a is present in b and `False' if
not.

    \begin{Verbatim}[commandchars=\\\{\}]
{\color{incolor}In [{\color{incolor}19}]:} \PY{l+s}{\PYZsq{}}\PY{l+s}{Fire}\PY{l+s}{\PYZsq{}} \PY{o+ow}{in} \PY{n}{names}
\end{Verbatim}

            \begin{Verbatim}[commandchars=\\\{\}]
{\color{outcolor}Out[{\color{outcolor}19}]:} True
\end{Verbatim}
        
    \begin{Verbatim}[commandchars=\\\{\}]
{\color{incolor}In [{\color{incolor}20}]:} \PY{l+s}{\PYZsq{}}\PY{l+s}{Rajath}\PY{l+s}{\PYZsq{}} \PY{o+ow}{in} \PY{n}{names}
\end{Verbatim}

            \begin{Verbatim}[commandchars=\\\{\}]
{\color{outcolor}Out[{\color{outcolor}20}]:} False
\end{Verbatim}
        
    In a list with elements as string, \textbf{max( )} and \textbf{min( )}
is applicable. \textbf{max( )} would return a string element whose ASCII
value is the highest and the lowest when \textbf{min( )} is used. Note
that only the first index of each element is considered each time and if
they value is the same then second index considered so on and so forth.

    \begin{Verbatim}[commandchars=\\\{\}]
{\color{incolor}In [{\color{incolor}21}]:} \PY{n}{mlist} \PY{o}{=} \PY{p}{[}\PY{l+s}{\PYZsq{}}\PY{l+s}{bzaa}\PY{l+s}{\PYZsq{}}\PY{p}{,}\PY{l+s}{\PYZsq{}}\PY{l+s}{ds}\PY{l+s}{\PYZsq{}}\PY{p}{,}\PY{l+s}{\PYZsq{}}\PY{l+s}{nc}\PY{l+s}{\PYZsq{}}\PY{p}{,}\PY{l+s}{\PYZsq{}}\PY{l+s}{az}\PY{l+s}{\PYZsq{}}\PY{p}{,}\PY{l+s}{\PYZsq{}}\PY{l+s}{z}\PY{l+s}{\PYZsq{}}\PY{p}{,}\PY{l+s}{\PYZsq{}}\PY{l+s}{klm}\PY{l+s}{\PYZsq{}}\PY{p}{]}
\end{Verbatim}

    \begin{Verbatim}[commandchars=\\\{\}]
{\color{incolor}In [{\color{incolor}22}]:} \PY{k}{print} \PY{n+nb}{max}\PY{p}{(}\PY{n}{mlist}\PY{p}{)}
         \PY{k}{print} \PY{n+nb}{min}\PY{p}{(}\PY{n}{mlist}\PY{p}{)}
\end{Verbatim}

    \begin{Verbatim}[commandchars=\\\{\}]
z
az
    \end{Verbatim}

    Here the first index of each element is considered and thus z has the
highest ASCII value thus it is returned and minimum ASCII is a. But what
if numbers are declared as strings?

    \begin{Verbatim}[commandchars=\\\{\}]
{\color{incolor}In [{\color{incolor}23}]:} \PY{n}{nlist} \PY{o}{=} \PY{p}{[}\PY{l+s}{\PYZsq{}}\PY{l+s}{1}\PY{l+s}{\PYZsq{}}\PY{p}{,}\PY{l+s}{\PYZsq{}}\PY{l+s}{94}\PY{l+s}{\PYZsq{}}\PY{p}{,}\PY{l+s}{\PYZsq{}}\PY{l+s}{93}\PY{l+s}{\PYZsq{}}\PY{p}{,}\PY{l+s}{\PYZsq{}}\PY{l+s}{1000}\PY{l+s}{\PYZsq{}}\PY{p}{]}
\end{Verbatim}

    \begin{Verbatim}[commandchars=\\\{\}]
{\color{incolor}In [{\color{incolor}24}]:} \PY{k}{print} \PY{n+nb}{max}\PY{p}{(}\PY{n}{nlist}\PY{p}{)}
         \PY{k}{print} \PY{n+nb}{min}\PY{p}{(}\PY{n}{nlist}\PY{p}{)}
\end{Verbatim}

    \begin{Verbatim}[commandchars=\\\{\}]
94
1
    \end{Verbatim}

    Even if the numbers are declared in a string the first index of each
element is considered and the maximum and minimum values are returned
accordingly.

    But if you want to find the \textbf{max( )} string element based on the
length of the string then another parameter `key=len' is declared inside
the \textbf{max( )} and \textbf{min( )} function.

    \begin{Verbatim}[commandchars=\\\{\}]
{\color{incolor}In [{\color{incolor}25}]:} \PY{k}{print} \PY{n+nb}{max}\PY{p}{(}\PY{n}{names}\PY{p}{,} \PY{n}{key}\PY{o}{=}\PY{n+nb}{len}\PY{p}{)}
         \PY{k}{print} \PY{n+nb}{min}\PY{p}{(}\PY{n}{names}\PY{p}{,} \PY{n}{key}\PY{o}{=}\PY{n+nb}{len}\PY{p}{)}
\end{Verbatim}

    \begin{Verbatim}[commandchars=\\\{\}]
Earth
Air
    \end{Verbatim}

    But even `Water' has length 5. \textbf{max()} or \textbf{min()} function
returns the first element when there are two or more elements with the
same length.

Any other built in function can be used or lambda function (will be
discussed later) in place of len.

A string can be converted into a list by using the \textbf{list()}
function.

    \begin{Verbatim}[commandchars=\\\{\}]
{\color{incolor}In [{\color{incolor}26}]:} \PY{n+nb}{list}\PY{p}{(}\PY{l+s}{\PYZsq{}}\PY{l+s}{hello}\PY{l+s}{\PYZsq{}}\PY{p}{)}
\end{Verbatim}

            \begin{Verbatim}[commandchars=\\\{\}]
{\color{outcolor}Out[{\color{outcolor}26}]:} ['h', 'e', 'l', 'l', 'o']
\end{Verbatim}
        
    \textbf{append( )} is used to add a element at the end of the list.

    \begin{Verbatim}[commandchars=\\\{\}]
{\color{incolor}In [{\color{incolor}27}]:} \PY{n}{lst} \PY{o}{=} \PY{p}{[}\PY{l+m+mi}{1}\PY{p}{,}\PY{l+m+mi}{1}\PY{p}{,}\PY{l+m+mi}{4}\PY{p}{,}\PY{l+m+mi}{8}\PY{p}{,}\PY{l+m+mi}{7}\PY{p}{]}
\end{Verbatim}

    \begin{Verbatim}[commandchars=\\\{\}]
{\color{incolor}In [{\color{incolor}28}]:} \PY{n}{lst}\PY{o}{.}\PY{n}{append}\PY{p}{(}\PY{l+m+mi}{1}\PY{p}{)}
         \PY{k}{print} \PY{n}{lst}
\end{Verbatim}

    \begin{Verbatim}[commandchars=\\\{\}]
[1, 1, 4, 8, 7, 1]
    \end{Verbatim}

    \textbf{count( )} is used to count the number of a particular element
that is present in the list.

    \begin{Verbatim}[commandchars=\\\{\}]
{\color{incolor}In [{\color{incolor}29}]:} \PY{n}{lst}\PY{o}{.}\PY{n}{count}\PY{p}{(}\PY{l+m+mi}{1}\PY{p}{)}
\end{Verbatim}

            \begin{Verbatim}[commandchars=\\\{\}]
{\color{outcolor}Out[{\color{outcolor}29}]:} 3
\end{Verbatim}
        
    \textbf{append( )} function can also be used to add a entire list at the
end. Observe that the resultant list becomes a nested list.

    \begin{Verbatim}[commandchars=\\\{\}]
{\color{incolor}In [{\color{incolor}30}]:} \PY{n}{lst1} \PY{o}{=} \PY{p}{[}\PY{l+m+mi}{5}\PY{p}{,}\PY{l+m+mi}{4}\PY{p}{,}\PY{l+m+mi}{2}\PY{p}{,}\PY{l+m+mi}{8}\PY{p}{]}
\end{Verbatim}

    \begin{Verbatim}[commandchars=\\\{\}]
{\color{incolor}In [{\color{incolor}31}]:} \PY{n}{lst}\PY{o}{.}\PY{n}{append}\PY{p}{(}\PY{n}{lst1}\PY{p}{)}
         \PY{k}{print} \PY{n}{lst}
\end{Verbatim}

    \begin{Verbatim}[commandchars=\\\{\}]
[1, 1, 4, 8, 7, 1, [5, 4, 2, 8]]
    \end{Verbatim}

    But if nested list is not what is desired then \textbf{extend( )}
function can be used.

    \begin{Verbatim}[commandchars=\\\{\}]
{\color{incolor}In [{\color{incolor}32}]:} \PY{n}{lst}\PY{o}{.}\PY{n}{extend}\PY{p}{(}\PY{n}{lst1}\PY{p}{)}
         \PY{k}{print} \PY{n}{lst}
\end{Verbatim}

    \begin{Verbatim}[commandchars=\\\{\}]
[1, 1, 4, 8, 7, 1, [5, 4, 2, 8], 5, 4, 2, 8]
    \end{Verbatim}

    \textbf{index( )} is used to find the index value of a particular
element. Note that if there are multiple elements of the same value then
the first index value of that element is returned.

    \begin{Verbatim}[commandchars=\\\{\}]
{\color{incolor}In [{\color{incolor}33}]:} \PY{n}{lst}\PY{o}{.}\PY{n}{index}\PY{p}{(}\PY{l+m+mi}{1}\PY{p}{)}
\end{Verbatim}

            \begin{Verbatim}[commandchars=\\\{\}]
{\color{outcolor}Out[{\color{outcolor}33}]:} 0
\end{Verbatim}
        
    \textbf{insert(x,y)} is used to insert a element y at a specified index
value x. \textbf{append( )} function made it only possible to insert at
the end.

    \begin{Verbatim}[commandchars=\\\{\}]
{\color{incolor}In [{\color{incolor}34}]:} \PY{n}{lst}\PY{o}{.}\PY{n}{insert}\PY{p}{(}\PY{l+m+mi}{5}\PY{p}{,} \PY{l+s}{\PYZsq{}}\PY{l+s}{name}\PY{l+s}{\PYZsq{}}\PY{p}{)}
         \PY{k}{print} \PY{n}{lst}
\end{Verbatim}

    \begin{Verbatim}[commandchars=\\\{\}]
[1, 1, 4, 8, 7, 'name', 1, [5, 4, 2, 8], 5, 4, 2, 8]
    \end{Verbatim}

    \textbf{insert(x,y)} inserts but does not replace element. If you want
to replace the element with another element you simply assign the value
to that particular index.

    \begin{Verbatim}[commandchars=\\\{\}]
{\color{incolor}In [{\color{incolor}35}]:} \PY{n}{lst}\PY{p}{[}\PY{l+m+mi}{5}\PY{p}{]} \PY{o}{=} \PY{l+s}{\PYZsq{}}\PY{l+s}{Python}\PY{l+s}{\PYZsq{}}
         \PY{k}{print} \PY{n}{lst}
\end{Verbatim}

    \begin{Verbatim}[commandchars=\\\{\}]
[1, 1, 4, 8, 7, 'Python', 1, [5, 4, 2, 8], 5, 4, 2, 8]
    \end{Verbatim}

    \textbf{pop( )} function return the last element in the list. This is
similar to the operation of a stack. Hence it wouldn't be wrong to tell
that lists can be used as a stack.

    \begin{Verbatim}[commandchars=\\\{\}]
{\color{incolor}In [{\color{incolor}36}]:} \PY{n}{lst}\PY{o}{.}\PY{n}{pop}\PY{p}{(}\PY{p}{)}
\end{Verbatim}

            \begin{Verbatim}[commandchars=\\\{\}]
{\color{outcolor}Out[{\color{outcolor}36}]:} 8
\end{Verbatim}
        
    Index value can be specified to pop a ceratin element corresponding to
that index value.

    \begin{Verbatim}[commandchars=\\\{\}]
{\color{incolor}In [{\color{incolor}37}]:} \PY{n}{lst}\PY{o}{.}\PY{n}{pop}\PY{p}{(}\PY{l+m+mi}{0}\PY{p}{)}
\end{Verbatim}

            \begin{Verbatim}[commandchars=\\\{\}]
{\color{outcolor}Out[{\color{outcolor}37}]:} 1
\end{Verbatim}
        
    \textbf{pop( )} is used to remove element based on it's index value
which can be assigned to a variable. One can also remove element by
specifying the element itself using the \textbf{remove( )} function.

    \begin{Verbatim}[commandchars=\\\{\}]
{\color{incolor}In [{\color{incolor}38}]:} \PY{n}{lst}\PY{o}{.}\PY{n}{remove}\PY{p}{(}\PY{l+s}{\PYZsq{}}\PY{l+s}{Python}\PY{l+s}{\PYZsq{}}\PY{p}{)}
         \PY{k}{print} \PY{n}{lst}
\end{Verbatim}

    \begin{Verbatim}[commandchars=\\\{\}]
[1, 4, 8, 7, 1, [5, 4, 2, 8], 5, 4, 2]
    \end{Verbatim}

    Alternative to \textbf{remove} function but with using index value is
\textbf{del}

    \begin{Verbatim}[commandchars=\\\{\}]
{\color{incolor}In [{\color{incolor}39}]:} \PY{k}{del} \PY{n}{lst}\PY{p}{[}\PY{l+m+mi}{1}\PY{p}{]}
         \PY{k}{print} \PY{n}{lst}
\end{Verbatim}

    \begin{Verbatim}[commandchars=\\\{\}]
[1, 8, 7, 1, [5, 4, 2, 8], 5, 4, 2]
    \end{Verbatim}

    The entire elements present in the list can be reversed by using the
\textbf{reverse()} function.

    \begin{Verbatim}[commandchars=\\\{\}]
{\color{incolor}In [{\color{incolor}40}]:} \PY{n}{lst}\PY{o}{.}\PY{n}{reverse}\PY{p}{(}\PY{p}{)}
         \PY{k}{print} \PY{n}{lst}
\end{Verbatim}

    \begin{Verbatim}[commandchars=\\\{\}]
[2, 4, 5, [5, 4, 2, 8], 1, 7, 8, 1]
    \end{Verbatim}

    Note that the nested list {[}5,4,2,8{]} is treated as a single element
of the parent list lst. Thus the elements inside the nested list is not
reversed.

Python offers built in operation \textbf{sort( )} to arrange the
elements in ascending order.

    \begin{Verbatim}[commandchars=\\\{\}]
{\color{incolor}In [{\color{incolor}41}]:} \PY{n}{lst}\PY{o}{.}\PY{n}{sort}\PY{p}{(}\PY{p}{)}
         \PY{k}{print} \PY{n}{lst}
\end{Verbatim}

    \begin{Verbatim}[commandchars=\\\{\}]
[1, 1, 2, 4, 5, 7, 8, [5, 4, 2, 8]]
    \end{Verbatim}

    For descending order, By default the reverse condition will be False for
reverse. Hence changing it to True would arrange the elements in
descending order.

    \begin{Verbatim}[commandchars=\\\{\}]
{\color{incolor}In [{\color{incolor}42}]:} \PY{n}{lst}\PY{o}{.}\PY{n}{sort}\PY{p}{(}\PY{n}{reverse}\PY{o}{=}\PY{n+nb+bp}{True}\PY{p}{)}
         \PY{k}{print} \PY{n}{lst}
\end{Verbatim}

    \begin{Verbatim}[commandchars=\\\{\}]
[[5, 4, 2, 8], 8, 7, 5, 4, 2, 1, 1]
    \end{Verbatim}

    Similarly for lists containing string elements, \textbf{sort( )} would
sort the elements based on it's ASCII value in ascending and by
specifying reverse=True in descending.

    \begin{Verbatim}[commandchars=\\\{\}]
{\color{incolor}In [{\color{incolor}43}]:} \PY{n}{names}\PY{o}{.}\PY{n}{sort}\PY{p}{(}\PY{p}{)}
         \PY{k}{print} \PY{n}{names}
         \PY{n}{names}\PY{o}{.}\PY{n}{sort}\PY{p}{(}\PY{n}{reverse}\PY{o}{=}\PY{n+nb+bp}{True}\PY{p}{)}
         \PY{k}{print} \PY{n}{names}
\end{Verbatim}

    \begin{Verbatim}[commandchars=\\\{\}]
['Air', 'Earth', 'Fire', 'Water']
['Water', 'Fire', 'Earth', 'Air']
    \end{Verbatim}

    To sort based on length key=len should be specified as shown.

    \begin{Verbatim}[commandchars=\\\{\}]
{\color{incolor}In [{\color{incolor}44}]:} \PY{n}{names}\PY{o}{.}\PY{n}{sort}\PY{p}{(}\PY{n}{key}\PY{o}{=}\PY{n+nb}{len}\PY{p}{)}
         \PY{k}{print} \PY{n}{names}
         \PY{n}{names}\PY{o}{.}\PY{n}{sort}\PY{p}{(}\PY{n}{key}\PY{o}{=}\PY{n+nb}{len}\PY{p}{,}\PY{n}{reverse}\PY{o}{=}\PY{n+nb+bp}{True}\PY{p}{)}
         \PY{k}{print} \PY{n}{names}
\end{Verbatim}

    \begin{Verbatim}[commandchars=\\\{\}]
['Air', 'Fire', 'Water', 'Earth']
['Water', 'Earth', 'Fire', 'Air']
    \end{Verbatim}

    \subsubsection{Copying a list}\label{copying-a-list}

    Most of the new python programmers commit this mistake. Consider the
following,

    \begin{Verbatim}[commandchars=\\\{\}]
{\color{incolor}In [{\color{incolor}45}]:} \PY{n}{lista}\PY{o}{=} \PY{p}{[}\PY{l+m+mi}{2}\PY{p}{,}\PY{l+m+mi}{1}\PY{p}{,}\PY{l+m+mi}{4}\PY{p}{,}\PY{l+m+mi}{3}\PY{p}{]}
\end{Verbatim}

    \begin{Verbatim}[commandchars=\\\{\}]
{\color{incolor}In [{\color{incolor}46}]:} \PY{n}{listb} \PY{o}{=} \PY{n}{lista}
         \PY{k}{print} \PY{n}{listb}
\end{Verbatim}

    \begin{Verbatim}[commandchars=\\\{\}]
[2, 1, 4, 3]
    \end{Verbatim}

    Here, We have declared a list, lista = {[}2,1,4,3{]}. This list is
copied to listb by assigning it's value and it get's copied as seen. Now
we perform some random operations on lista.

    \begin{Verbatim}[commandchars=\\\{\}]
{\color{incolor}In [{\color{incolor}47}]:} \PY{n}{lista}\PY{o}{.}\PY{n}{pop}\PY{p}{(}\PY{p}{)}
         \PY{k}{print} \PY{n}{lista}
         \PY{n}{lista}\PY{o}{.}\PY{n}{append}\PY{p}{(}\PY{l+m+mi}{9}\PY{p}{)}
         \PY{k}{print} \PY{n}{lista}
\end{Verbatim}

    \begin{Verbatim}[commandchars=\\\{\}]
[2, 1, 4]
[2, 1, 4, 9]
    \end{Verbatim}

    \begin{Verbatim}[commandchars=\\\{\}]
{\color{incolor}In [{\color{incolor}48}]:} \PY{k}{print} \PY{n}{listb}
\end{Verbatim}

    \begin{Verbatim}[commandchars=\\\{\}]
[2, 1, 4, 9]
    \end{Verbatim}

    listb has also changed though no operation has been performed on it.
This is because you have assigned the same memory space of lista to
listb. So how do fix this?

If you recall, in slicing we had seen that parentlist{[}a:b{]} returns a
list from parent list with start index a and end index b and if a and b
is not mentioned then by default it considers the first and last
element. We use the same concept here. By doing so, we are assigning the
data of lista to listb as a variable.

    \begin{Verbatim}[commandchars=\\\{\}]
{\color{incolor}In [{\color{incolor}49}]:} \PY{n}{lista} \PY{o}{=} \PY{p}{[}\PY{l+m+mi}{2}\PY{p}{,}\PY{l+m+mi}{1}\PY{p}{,}\PY{l+m+mi}{4}\PY{p}{,}\PY{l+m+mi}{3}\PY{p}{]}
\end{Verbatim}

    \begin{Verbatim}[commandchars=\\\{\}]
{\color{incolor}In [{\color{incolor}50}]:} \PY{n}{listb} \PY{o}{=} \PY{n}{lista}\PY{p}{[}\PY{p}{:}\PY{p}{]}
         \PY{k}{print} \PY{n}{listb}
\end{Verbatim}

    \begin{Verbatim}[commandchars=\\\{\}]
[2, 1, 4, 3]
    \end{Verbatim}

    \begin{Verbatim}[commandchars=\\\{\}]
{\color{incolor}In [{\color{incolor}51}]:} \PY{n}{lista}\PY{o}{.}\PY{n}{pop}\PY{p}{(}\PY{p}{)}
         \PY{k}{print} \PY{n}{lista}
         \PY{n}{lista}\PY{o}{.}\PY{n}{append}\PY{p}{(}\PY{l+m+mi}{9}\PY{p}{)}
         \PY{k}{print} \PY{n}{lista}
\end{Verbatim}

    \begin{Verbatim}[commandchars=\\\{\}]
[2, 1, 4]
[2, 1, 4, 9]
    \end{Verbatim}

    \begin{Verbatim}[commandchars=\\\{\}]
{\color{incolor}In [{\color{incolor}52}]:} \PY{k}{print} \PY{n}{listb}
\end{Verbatim}

    \begin{Verbatim}[commandchars=\\\{\}]
[2, 1, 4, 3]
    \end{Verbatim}

    \subsection{Tuples}\label{tuples}

    Tuples are similar to lists but only big difference is the elements
inside a list can be changed but in tuple it cannot be changed. Think of
tuples as something which has to be True for a particular something and
cannot be True for no other values. For better understanding, Recall
\textbf{divmod()} function.

    \begin{Verbatim}[commandchars=\\\{\}]
{\color{incolor}In [{\color{incolor}53}]:} \PY{n}{xyz} \PY{o}{=} \PY{n+nb}{divmod}\PY{p}{(}\PY{l+m+mi}{10}\PY{p}{,}\PY{l+m+mi}{3}\PY{p}{)}
         \PY{k}{print} \PY{n}{xyz}
         \PY{k}{print} \PY{n+nb}{type}\PY{p}{(}\PY{n}{xyz}\PY{p}{)}
\end{Verbatim}

    \begin{Verbatim}[commandchars=\\\{\}]
(3, 1)
<type 'tuple'>
    \end{Verbatim}

    Here the quotient has to be 3 and the remainder has to be 1. These
values cannot be changed whatsoever when 10 is divided by 3. Hence
divmod returns these values in a tuple.

    To define a tuple, A variable is assigned to paranthesis ( ) or tuple(
).

    \begin{Verbatim}[commandchars=\\\{\}]
{\color{incolor}In [{\color{incolor}54}]:} \PY{n}{tup} \PY{o}{=} \PY{p}{(}\PY{p}{)}
         \PY{n}{tup2} \PY{o}{=} \PY{n+nb}{tuple}\PY{p}{(}\PY{p}{)}
\end{Verbatim}

    If you want to directly declare a tuple it can be done by using a comma
at the end of the data.

    \begin{Verbatim}[commandchars=\\\{\}]
{\color{incolor}In [{\color{incolor}55}]:} \PY{l+m+mi}{27}\PY{p}{,}
\end{Verbatim}

            \begin{Verbatim}[commandchars=\\\{\}]
{\color{outcolor}Out[{\color{outcolor}55}]:} (27,)
\end{Verbatim}
        
    27 when multiplied by 2 yields 54, But when multiplied with a tuple the
data is repeated twice.

    \begin{Verbatim}[commandchars=\\\{\}]
{\color{incolor}In [{\color{incolor}56}]:} \PY{l+m+mi}{2}\PY{o}{*}\PY{p}{(}\PY{l+m+mi}{27}\PY{p}{,}\PY{p}{)}
\end{Verbatim}

            \begin{Verbatim}[commandchars=\\\{\}]
{\color{outcolor}Out[{\color{outcolor}56}]:} (27, 27)
\end{Verbatim}
        
    Values can be assigned while declaring a tuple. It takes a list as input
and converts it into a tuple or it takes a string and converts it into a
tuple.

    \begin{Verbatim}[commandchars=\\\{\}]
{\color{incolor}In [{\color{incolor}57}]:} \PY{n}{tup3} \PY{o}{=} \PY{n+nb}{tuple}\PY{p}{(}\PY{p}{[}\PY{l+m+mi}{1}\PY{p}{,}\PY{l+m+mi}{2}\PY{p}{,}\PY{l+m+mi}{3}\PY{p}{]}\PY{p}{)}
         \PY{k}{print} \PY{n}{tup3}
         \PY{n}{tup4} \PY{o}{=} \PY{n+nb}{tuple}\PY{p}{(}\PY{l+s}{\PYZsq{}}\PY{l+s}{Hello}\PY{l+s}{\PYZsq{}}\PY{p}{)}
         \PY{k}{print} \PY{n}{tup4}
\end{Verbatim}

    \begin{Verbatim}[commandchars=\\\{\}]
(1, 2, 3)
('H', 'e', 'l', 'l', 'o')
    \end{Verbatim}

    It follows the same indexing and slicing as Lists.

    \begin{Verbatim}[commandchars=\\\{\}]
{\color{incolor}In [{\color{incolor}58}]:} \PY{k}{print} \PY{n}{tup3}\PY{p}{[}\PY{l+m+mi}{1}\PY{p}{]}
         \PY{n}{tup5} \PY{o}{=} \PY{n}{tup4}\PY{p}{[}\PY{p}{:}\PY{l+m+mi}{3}\PY{p}{]}
         \PY{k}{print} \PY{n}{tup5}
\end{Verbatim}

    \begin{Verbatim}[commandchars=\\\{\}]
2
('H', 'e', 'l')
    \end{Verbatim}

    \subsubsection{Mapping one tuple to
another}\label{mapping-one-tuple-to-another}

    \begin{Verbatim}[commandchars=\\\{\}]
{\color{incolor}In [{\color{incolor}59}]:} \PY{p}{(}\PY{n}{a}\PY{p}{,}\PY{n}{b}\PY{p}{,}\PY{n}{c}\PY{p}{)}\PY{o}{=} \PY{p}{(}\PY{l+s}{\PYZsq{}}\PY{l+s}{alpha}\PY{l+s}{\PYZsq{}}\PY{p}{,}\PY{l+s}{\PYZsq{}}\PY{l+s}{beta}\PY{l+s}{\PYZsq{}}\PY{p}{,}\PY{l+s}{\PYZsq{}}\PY{l+s}{gamma}\PY{l+s}{\PYZsq{}}\PY{p}{)}
\end{Verbatim}

    \begin{Verbatim}[commandchars=\\\{\}]
{\color{incolor}In [{\color{incolor}60}]:} \PY{k}{print} \PY{n}{a}\PY{p}{,}\PY{n}{b}\PY{p}{,}\PY{n}{c}
\end{Verbatim}

    \begin{Verbatim}[commandchars=\\\{\}]
alpha beta gamma
    \end{Verbatim}

    \begin{Verbatim}[commandchars=\\\{\}]
{\color{incolor}In [{\color{incolor}61}]:} \PY{n}{d} \PY{o}{=} \PY{n+nb}{tuple}\PY{p}{(}\PY{l+s}{\PYZsq{}}\PY{l+s}{RajathKumarMP}\PY{l+s}{\PYZsq{}}\PY{p}{)}
         \PY{k}{print} \PY{n}{d}
\end{Verbatim}

    \begin{Verbatim}[commandchars=\\\{\}]
('R', 'a', 'j', 'a', 't', 'h', 'K', 'u', 'm', 'a', 'r', 'M', 'P')
    \end{Verbatim}

    \subsubsection{Built In Tuple functions}\label{built-in-tuple-functions}

    \textbf{count()} function counts the number of specified element that is
present in the tuple.

    \begin{Verbatim}[commandchars=\\\{\}]
{\color{incolor}In [{\color{incolor}62}]:} \PY{n}{d}\PY{o}{.}\PY{n}{count}\PY{p}{(}\PY{l+s}{\PYZsq{}}\PY{l+s}{a}\PY{l+s}{\PYZsq{}}\PY{p}{)}
\end{Verbatim}

            \begin{Verbatim}[commandchars=\\\{\}]
{\color{outcolor}Out[{\color{outcolor}62}]:} 3
\end{Verbatim}
        
    \textbf{index()} function returns the index of the specified element. If
the elements are more than one then the index of the first element of
that specified element is returned

    \begin{Verbatim}[commandchars=\\\{\}]
{\color{incolor}In [{\color{incolor}63}]:} \PY{n}{d}\PY{o}{.}\PY{n}{index}\PY{p}{(}\PY{l+s}{\PYZsq{}}\PY{l+s}{a}\PY{l+s}{\PYZsq{}}\PY{p}{)}
\end{Verbatim}

            \begin{Verbatim}[commandchars=\\\{\}]
{\color{outcolor}Out[{\color{outcolor}63}]:} 1
\end{Verbatim}
        
    \subsection{Sets}\label{sets}

    Sets are mainly used to eliminate repeated numbers in a sequence/list.
It is also used to perform some standard set operations.

Sets are declared as set() which will initialize a empty set. Also
set({[}sequence{]}) can be executed to declare a set with elements

    \begin{Verbatim}[commandchars=\\\{\}]
{\color{incolor}In [{\color{incolor}64}]:} \PY{n}{set1} \PY{o}{=} \PY{n+nb}{set}\PY{p}{(}\PY{p}{)}
         \PY{k}{print} \PY{n+nb}{type}\PY{p}{(}\PY{n}{set1}\PY{p}{)}
\end{Verbatim}

    \begin{Verbatim}[commandchars=\\\{\}]
<type 'set'>
    \end{Verbatim}

    \begin{Verbatim}[commandchars=\\\{\}]
{\color{incolor}In [{\color{incolor}65}]:} \PY{n}{set0} \PY{o}{=} \PY{n+nb}{set}\PY{p}{(}\PY{p}{[}\PY{l+m+mi}{1}\PY{p}{,}\PY{l+m+mi}{2}\PY{p}{,}\PY{l+m+mi}{2}\PY{p}{,}\PY{l+m+mi}{3}\PY{p}{,}\PY{l+m+mi}{3}\PY{p}{,}\PY{l+m+mi}{4}\PY{p}{]}\PY{p}{)}
         \PY{k}{print} \PY{n}{set0}
\end{Verbatim}

    \begin{Verbatim}[commandchars=\\\{\}]
set([1, 2, 3, 4])
    \end{Verbatim}

    elements 2,3 which are repeated twice are seen only once. Thus in a set
each element is distinct.

    \subsubsection{Built-in Functions}\label{built-in-functions}

    \begin{Verbatim}[commandchars=\\\{\}]
{\color{incolor}In [{\color{incolor}66}]:} \PY{n}{set1} \PY{o}{=} \PY{n+nb}{set}\PY{p}{(}\PY{p}{[}\PY{l+m+mi}{1}\PY{p}{,}\PY{l+m+mi}{2}\PY{p}{,}\PY{l+m+mi}{3}\PY{p}{]}\PY{p}{)}
\end{Verbatim}

    \begin{Verbatim}[commandchars=\\\{\}]
{\color{incolor}In [{\color{incolor}67}]:} \PY{n}{set2} \PY{o}{=} \PY{n+nb}{set}\PY{p}{(}\PY{p}{[}\PY{l+m+mi}{2}\PY{p}{,}\PY{l+m+mi}{3}\PY{p}{,}\PY{l+m+mi}{4}\PY{p}{,}\PY{l+m+mi}{5}\PY{p}{]}\PY{p}{)}
\end{Verbatim}

    \textbf{union( )} function returns a set which contains all the elements
of both the sets without repition.

    \begin{Verbatim}[commandchars=\\\{\}]
{\color{incolor}In [{\color{incolor}68}]:} \PY{n}{set1}\PY{o}{.}\PY{n}{union}\PY{p}{(}\PY{n}{set2}\PY{p}{)}
\end{Verbatim}

            \begin{Verbatim}[commandchars=\\\{\}]
{\color{outcolor}Out[{\color{outcolor}68}]:} \{1, 2, 3, 4, 5\}
\end{Verbatim}
        
    \textbf{add( )} will add a particular element into the set. Note that
the index of the newly added element is arbitrary and can be placed
anywhere not neccessarily in the end.

    \begin{Verbatim}[commandchars=\\\{\}]
{\color{incolor}In [{\color{incolor}69}]:} \PY{n}{set1}\PY{o}{.}\PY{n}{add}\PY{p}{(}\PY{l+m+mi}{0}\PY{p}{)}
         \PY{n}{set1}
\end{Verbatim}

            \begin{Verbatim}[commandchars=\\\{\}]
{\color{outcolor}Out[{\color{outcolor}69}]:} \{0, 1, 2, 3\}
\end{Verbatim}
        
    \textbf{intersection( )} function outputs a set which contains all the
elements that are in both sets.

    \begin{Verbatim}[commandchars=\\\{\}]
{\color{incolor}In [{\color{incolor}70}]:} \PY{n}{set1}\PY{o}{.}\PY{n}{intersection}\PY{p}{(}\PY{n}{set2}\PY{p}{)}
\end{Verbatim}

            \begin{Verbatim}[commandchars=\\\{\}]
{\color{outcolor}Out[{\color{outcolor}70}]:} \{2, 3\}
\end{Verbatim}
        
    \textbf{difference( )} function ouptuts a set which contains elements
that are in set1 and not in set2.

    \begin{Verbatim}[commandchars=\\\{\}]
{\color{incolor}In [{\color{incolor}71}]:} \PY{n}{set1}\PY{o}{.}\PY{n}{difference}\PY{p}{(}\PY{n}{set2}\PY{p}{)}
\end{Verbatim}

            \begin{Verbatim}[commandchars=\\\{\}]
{\color{outcolor}Out[{\color{outcolor}71}]:} \{0, 1\}
\end{Verbatim}
        
    \textbf{symmetric\_difference( )} function ouputs a function which
contains elements that are in one of the sets.

    \begin{Verbatim}[commandchars=\\\{\}]
{\color{incolor}In [{\color{incolor}72}]:} \PY{n}{set2}\PY{o}{.}\PY{n}{symmetric\PYZus{}difference}\PY{p}{(}\PY{n}{set1}\PY{p}{)}
\end{Verbatim}

            \begin{Verbatim}[commandchars=\\\{\}]
{\color{outcolor}Out[{\color{outcolor}72}]:} \{0, 1, 4, 5\}
\end{Verbatim}
        
    \textbf{issubset( ), isdisjoint( ), issuperset( )} is used to check if
the set1/set2 is a subset, disjoint or superset of set2/set1
respectively.

    \begin{Verbatim}[commandchars=\\\{\}]
{\color{incolor}In [{\color{incolor}73}]:} \PY{n}{set1}\PY{o}{.}\PY{n}{issubset}\PY{p}{(}\PY{n}{set2}\PY{p}{)}
\end{Verbatim}

            \begin{Verbatim}[commandchars=\\\{\}]
{\color{outcolor}Out[{\color{outcolor}73}]:} False
\end{Verbatim}
        
    \begin{Verbatim}[commandchars=\\\{\}]
{\color{incolor}In [{\color{incolor}74}]:} \PY{n}{set2}\PY{o}{.}\PY{n}{isdisjoint}\PY{p}{(}\PY{n}{set1}\PY{p}{)}
\end{Verbatim}

            \begin{Verbatim}[commandchars=\\\{\}]
{\color{outcolor}Out[{\color{outcolor}74}]:} False
\end{Verbatim}
        
    \begin{Verbatim}[commandchars=\\\{\}]
{\color{incolor}In [{\color{incolor}75}]:} \PY{n}{set2}\PY{o}{.}\PY{n}{issuperset}\PY{p}{(}\PY{n}{set1}\PY{p}{)}
\end{Verbatim}

            \begin{Verbatim}[commandchars=\\\{\}]
{\color{outcolor}Out[{\color{outcolor}75}]:} False
\end{Verbatim}
        
    \textbf{pop( )} is used to remove an arbitrary element in the set

    \begin{Verbatim}[commandchars=\\\{\}]
{\color{incolor}In [{\color{incolor}76}]:} \PY{n}{set1}\PY{o}{.}\PY{n}{pop}\PY{p}{(}\PY{p}{)}
         \PY{k}{print} \PY{n}{set1}
\end{Verbatim}

    \begin{Verbatim}[commandchars=\\\{\}]
set([1, 2, 3])
    \end{Verbatim}

    \textbf{remove( )} function deletes the specified element from the set.

    \begin{Verbatim}[commandchars=\\\{\}]
{\color{incolor}In [{\color{incolor}77}]:} \PY{n}{set1}\PY{o}{.}\PY{n}{remove}\PY{p}{(}\PY{l+m+mi}{2}\PY{p}{)}
         \PY{n}{set1}
\end{Verbatim}

            \begin{Verbatim}[commandchars=\\\{\}]
{\color{outcolor}Out[{\color{outcolor}77}]:} \{1, 3\}
\end{Verbatim}
        
    \textbf{clear( )} is used to clear all the elements and make that set an
empty set.

    \begin{Verbatim}[commandchars=\\\{\}]
{\color{incolor}In [{\color{incolor}78}]:} \PY{n}{set1}\PY{o}{.}\PY{n}{clear}\PY{p}{(}\PY{p}{)}
         \PY{n}{set1}
\end{Verbatim}

            \begin{Verbatim}[commandchars=\\\{\}]
{\color{outcolor}Out[{\color{outcolor}78}]:} set()
\end{Verbatim}
        

    % Add a bibliography block to the postdoc
  \newpage
  
  \input{04}  
    
    
    \end{document}
  
    
    
    \end{document}

    
    \end{document}

    
    
    \end{document}
